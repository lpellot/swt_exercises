\documentclass[12pt]{exam}
\usepackage[latin1]{inputenc}

\usepackage{verbatim}
\usepackage[margin=1in]{geometry}
\usepackage{amsmath,amssymb}
\usepackage{multicol}
\usepackage[ngerman]{babel}
\usepackage{tikz}
\usetikzlibrary{arrows, petri, positioning, shapes}

\usepackage{listings}
\lstset{
    basicstyle = \ttfamily \footnotesize,
    frame = single,
    keepspaces = true,
    numbers = left,
}

\usepackage{fontspec}
\setmainfont{Noto Sans}

\newcommand{\class}{Softwaretechnik}
\newcommand{\term}{Aufgabensammlung}
\newcommand{\examnum}{}
\newcommand{\examdate}{Stand: \today{}}
\newcommand{\timelimit}{Gibts nicht}

\pagestyle{head}
\firstpageheader{}{}{}
\runningheader{\class}{\examnum\ - Page \thepage\ of \numpages}{\examdate}
\runningheadrule


\begin{document}

% todo
% 20140508_103121

\noindent
\begin{tabular*}{\textwidth}{l @{\extracolsep{\fill}} r @{\extracolsep{6pt}} l}
\textbf{\class} & \textbf{Name:} & \makebox[2in]{\hrulefill}\\
\textbf{\term} &&\\
\textbf{\examnum} &&\\
\textbf{\examdate} &&\\
\textbf{Time Limit: \timelimit} & Teaching Assistant & \makebox[2in]{\hrulefill}
\end{tabular*}\\
\rule[2ex]{\textwidth}{2pt}

This exam contains \numpages\ pages (including this cover page) and \numquestions\ questions.\\
Total of points is \numpoints.

Rest of introduction. Rest of introduction. Rest of introduction. Rest of introduction. Rest of introduction. Rest of introduction. Rest of introduction. Rest of introduction. 

\begin{comment}
\begin{center}
Grade Table (for teacher use only)\\
\addpoints
\gradetable[v][questions]
\end{center}
\end{comment}

\noindent
\rule[2ex]{\textwidth}{2pt}

\begin{questions}

\question[5] Geben sie durch Ankreuzen an eines der Antwortfelder an, ob die folgenden Aussagen wahr oder falsch sind. \\
\addpoints
\begin{tabular}{| p{12cm} | c | c |} \hline
\textbf{Aussage} & \textbf{Wahr} & \textbf{Falsch} \\ \hline
Auf den Feinentwurf kann man verzichten, auf den Implementierungsentwurf jedoch nicht. & & \\ \hline
Kleine Kohäsion und große Kopplung sind Zeichen eines guten Entwurfs. & & \\ \hline
Eine gute Modularisierung ist eine wichtige Voraussetzung für eine arbeitsteilige Implementierung. & & \\ \hline
Das Transformationszentrum in einem Datenflussdiagramm ergibt sich rein aus dessen Struktur. & & \\ \hline
Ein Verteilungsdiagramm zeigt die Kommunikationsverbindungen und Abhängigkeiten von zur Laufzeit physisch vorhandenen Ressourcen. & & \\ \hline
\end{tabular}

\question[2] Nennen sie \textbf{zwei Vorteile von Entwurfsmustern}.
\addpoints

\question[2] In der unteren Abbildung ist ein schematisches DFD dargestellt. Woran können Sie erkennen, ob es sich bei dem Prozess \textbf{D} um ein Transformationszentrum oder ein Transaktionszentrum handelt? \\
\addpoints
\begin{tikzpicture}
    \matrix[row sep = 1cm, column sep = 1cm]{
        % 1st row
        \node[place, white] (00) {}; &
        \node[place] (A) {A}; &
        \node[place] (B) {B}; &
        \node[place, white] (03) {}; &
        \node[place] (E) {E}; &
        \node[place] (F) {F}; &
        \node[place, white] (06) {}; \\
        % 2nd row
        \node[place, white] (10) {}; &
        \node[place, white] (11) {}; &
        \node[place] (C) {C}; &
        \node[place] (D) {D}; &
        \node[place] (G) {G}; &
        \node[place, white] (15) {}; &
        \node[place, white] (16) {}; \\
        % 3rd row
        \node[place, white] (20) {}; &
        \node[place, white] (21) {}; &
        \node[place, white] (22) {}; &
        \node[place, white] (23) {}; &
        \node[place] (H) {H}; &
        \node[place, white] (25) {}; &
        \node[place, white] (26) {}; \\
    };
    
    \draw[->] (00) -- (A);
    \draw[->] (A) -- (B);
    \draw[->] (B) -- (D);
    \draw[->] (E) -- (F);
    \draw[->] (F) -- (06);
    \draw[->] (11) -- (C);
    \draw[->] (C) -- (D);
    \draw[->] (D) -- (E);
    \draw[->] (D) -- (G);
    \draw[->] (G) -- (15);
    \draw[->] (D) -- (H);
    \draw[->] (H) -- (25);
\end{tikzpicture}

\question[2] \textbf{Erklären} Sie die \textbf{Merkmale der Strukturierten Methoden}.
\addpoints

\question[3] \textbf{Erklären} Sie die \textbf{Bedeutung} folgender Notationselemente aus einem \textbf{Datendlussdiagramm} in jeweils einem Satz.
\addpoints
\begin{parts}
    \part \tikz \draw (0, 0) rectangle (2, 1);
    \part \tikz \draw (0, 0) circle (5mm);
    \part \tikz \draw[->] (0, 0) -- (2, 0);
\end{parts}

\question[2] Wie können bei der \textbf{Strukturierten Analyse} (SA) Prozesse beschrieben werden?

\question[4]
\noaddpoints
\begin{parts}
    \part[2] Erläutern sie kurz den Unterschied zwischen einem \textbf{Programm} und einem \textbf{Softwaresystem}.
    \part[1] \textbf{Kreuzen} Sie \textbf{Ihre Antwort an}. \\Wie hoch ist die durchschnittliche Produktivität pro Person bei sicherheitskritischer Software (inklusive Test und Dokumentation) in \emph{Zeilen ausführbares Programm pro Tag}?
    \begin{checkboxes}
        \choice 50 Zeilen pro Tag
        \choice 25 Zeilen pro Tag
        \choice 15 Zeilen pro Tag
        \choice 2 Zeilen pro Tag
    \end{checkboxes}
    \part[1] \textbf{Kreuzen} Sie {Ihre Antwort an}. \\Wie viel Prozent der Programmierer in Unternehmen sind laut Vorlesung durchschnittlich mit der Wartung und Verbesserung von altem Code beschäftigt?
    \begin{checkboxes}
        \choice $<10\%$
        \choice $10-29\%$
        \choice $30-50\%$
        \choice $51-70\%$
        \choice $>70\%$
    \end{checkboxes}
\end{parts}
\addpoints

\question[8] \textbf{Beantworten} Sie die folgenden Fragen. 
\noaddpoints
\begin{parts}
    \part[2] \textbf{Erklären} Sie in Stichworten den Unterschied zwischen einem \textbf{präskriptiven} und einem \textbf{deskriptiven} Modell.
    \part[2] \textbf{Nennen} Sie \textbf{vier gebräuchliche Darstellungsformen für Modelle} im Software Engineering.
    \part[2] \textbf{Nennen} sie \textbf{vier Aspekte}, die durch den \textbf{Einsatz von Modellen verbessert} werden können.
    \part[2] Die Abbildung zeigt die Unterstützung einer riskanten Modifikation durch Modelle. \textbf{Ergänzen} sie die Abbildung um die fehlenden Begriffe. \\
    \begin{tikzpicture}[node distance = 3 cm]
        \node[place] (00) {}
            node[xshift = -20mm] {Ist-Zustand};
        \node[place, right = of 00] (01) {}
            node[xshift = 70mm, yshift = -3mm]{\underline{\hspace{40mm}}};
        \node[place, below = of 00] (10) {}
            node[xshift = -20mm, yshift = -40mm, align = center] {Geplanter \\Zustand};
        \node[place, right = of 10] (11) {}
            node[xshift = 70mm, yshift = -43mm] {\underline{\hspace{40mm}}};
        
        \draw[->] (00) -- (01)
            node[midway, fill = white, yshift = -4mm] {Modellierung};
        \draw[->] (00) -- (10)
            node[midway, fill = white, align = center, xshift = -15mm] {Riskante \\Modifikation};
        \draw[->] (01) -- (11)
            node[midway, xshift = 30mm, yshift = -3mm] {\underline{\hspace{40mm}}};
        \draw[->] (11) -- (10);

    \end{tikzpicture}
\end{parts}
\addpoints

\question[3] Softwaregestützte Systeme können nach Anwendungsgebiet grob in drei Klassen (z. B. "`technische Systeme"') eingeordnet werden. \textbf{Ordnen} Sie die folgenden Beispiele jeweils der \textbf{richtigen Klasse} zu.
\addpoints
{
\checkboxchar{\underline{\hspace{4cm}}}
\begin{checkboxes}
\choice Bubblesort
\choice Flugleitsystem
\choice Wettervorhersage
\choice Autopilot
\choice Tetris
\choice Facebook
\end{checkboxes}
}

\question[2] \textbf{Nennen} Sie \textbf{zwei Wechselwirkungen} zwischen System und Systemumgebung.
\addpoints

\question[3] Tragen Sie in die Abbildung die \textbf{Bezeichnungen der Phasen des System Engineering-Prozesses} ein. \\
\addpoints
\begin{tikzpicture}[transition/.style={minimum height = 1.5cm, minimum width = 3cm, draw = black}]
    \node[transition] (00) {};
    \node[transition, xshift = 4cm, yshift = 0cm] (01) {};
    \node[transition, xshift = 8cm, yshift = 0cm] (02) {};
    \node[transition, xshift = 12cm, yshift = 0cm] (03) {};
    
    \node[transition, xshift = 4cm, yshift = -2.5cm] (11) {Ausmusterung};
    \node[transition, xshift = 8cm, yshift = -2.5cm] (12) {};
    \node[transition, xshift = 12cm, yshift = -2.5cm] (13) {};

    \draw[->] (00) -- (01);
    \draw[->] (01) -- (02);
    \draw[->] (02) -- (03);
    \draw[->] (03) -- (13);
    \draw[->] (13) -- (12);
    \draw[->] (12) -- (11);
\end{tikzpicture}

\question[2] Struktur und Merkmale von im Prozess benötigten Dokumenten gehören zu den Inhalten eines Prozessmodells. \textbf{Nennen} Sie \textbf{vier weitere Inhalte} eines Prozessmodells.
\addpoints

\question[3] Zu einem Prozessmodell gehören beispielsweise Prüf- und Fertigstellungskriterien. \textbf{Nennen} Sie \textbf{drei weitere Inhalte} eines Prozessmodells.
\addpoints

\question[2] Ordnen Sie das \textbf{Wasserfallmodell} ein, indem Sie die fehlenden Elemente in der Abbildung ergänzen. \\
\addpoints
\begin{tikzpicture}[grow = right,
                    level 1/.style={sibling distance = 6em},
                    level 2/.style={sibling distance = 3em},
                    level distance = 4cm,
                    transition/.style={text width = 3cm, minimum height = 1cm, draw = black}]
    \node[transition]{Vorgehens- und Prozessmodelle}
    child{node[transition]{}}
    child{node[transition]{}}
    child{node[transition]{}
        child{node[transition]{}}
        child{node[transition]{}
            child{node[transition]{}}
        }
    };
\end{tikzpicture}

\question[2] Nennen sie zwei nicht-lineare Prozessmodelle.
\addpoints

\question[3] \textbf{Nennen} Sie \textbf{zwei Vorteile} und \textbf{einen Nachteil} des \textbf{klassischen Wasserfallmodells}.
\addpoints
{
\checkboxchar{ }
\begin{checkboxes}
    \choice \textbf{+}
    \choice \textbf{+}
    \choice \textbf{-}
\end{checkboxes}
}

\question[4] \textbf{Erläutern} Sie die Begriffe \textbf{Verifikation} und \textbf{Validation}.
\addpoints

\question[2] Systemerstellung (SE)  ist ein Submodell des V-Modell 97. Nennen Sie zwei weitere Submodelle des V-Modells 97.
\addpoints

\question[2] Ordnen Sie das V-Modell 97 in den \textbf{Kontext der Vorgehens- und Prozessmodelle} ein.
\addpoints

\question[2] Geben sie durch Ankreuzen des jeweiligen Feldes an, ob die folgenden Aussagen wahr oder falsch sind.
\addpoints
\begin{tabular}{| p{12cm} | c | c |}
    Zu den Submodellen des V-Modells geh"ort das Werkzeugmanagement. & & \\ \hline
    Das Spiralmodell ist ein generisches Modell f"ur Projektmanagement. & & \\ \hline
\end{tabular}

\question[3] Die Abbildung zeigt die Aufwandsverteilung in Abh"angigkeit der Projektgr"o"se. \\
\noaddpoints
\begin{tikzpicture}
    \begin{scope}[xshift = 1cm]
        \draw (0, 0) -- (8, 0) -- (8, 2) -- (0, 1) -- cycle;
        \draw (0, 1) -- (8, 2) node[xshift = -2cm, yshift = 1.5cm] {Dokumentation} -- (8, 2.5) -- (0, 4) -- cycle;
        \draw (0, 4) -- (8, 2.5) -- (8, 4) -- (0, 4.5) -- cycle;
        \draw (0, 4.5) -- (8, 4) -- (8, 5) -- (0, 5) -- cycle;
    \end{scope}

    %TODO: x axis blabla
    \foreach \y/\ytext in {0/$0\%$, .5/$10\%$, 1/$20\%$, 1.5/$30\%$, 2/$40\%$, 2.5/$50\%$, 3/$60\%$, 3.5/$70\%$, 4/$80\%$, 4.5/$90\%$, 5/$100\%$}
        \node[yshift = \y cm, font = \footnotesize] {\ytext};
\end{tikzpicture}
\begin{parts}
    \part[1.5] Tragen Sie die fehlenden Aktivit"aten \textbf{Management, Codierung und Qualit"atssicherung} in die Abbildung ein.
    \part[1.5] Begr"unden Sie in \textbf{ganzen S"atzen}, warum sich die Aufwandsverteilung so verh"alt.
\end{parts}
\addpoints

\question[6] Das \textbf{V-Modell 97} legt Aktivit"aten, Produkte und Zust"ande des Entwicklungsprozesses fest. Tragen sie die \textbf{Aktivit"aten} des \textbf{Submodells f"ur Systemerstellung (SE)} in die Abbildung ein.\\
\addpoints
\begin{tikzpicture}[rectangle/.style={draw = black, fill = white, minimum width = 3cm, minimum height = 1cm}]
    \draw (0, 0) -- (5, -10) -- (10, 0);
    \foreach \xshift/\yshift in {0, 1/2, 2/4, 3/6, 4/8, 5/10, 10/0, 9/2, 8/4}
        \node[xshift = \xshift cm, yshift = -\yshift cm, rectangle] {};
\end{tikzpicture}

\question[10]
\noaddpoints
\begin{parts}
    \part[6] Vergleichen Sie das \textbf{V-Modell 97} mit dem Prozessmodell \textbf{Prototyping}. Beziehen Sie in der Verglech auch die Vor- und Nachteile beider Modelle mit ein.
    \part[4] Nennen Sie \textbf{zwei weitere Prozessmodelle} und beschreiben Sie jedes in ein bis zwei S"atzen.
\end{parts}
\addpoints

\question[4] Geben Sie durch Ankreuzen des jeweiligen Antwortfeldes an, ob die folgenden Aussagen wahr oder falsch sind. \\
\addpoints
\begin{tabular}{| p{12cm} | c | c |} \hline
    \textbf{Aussage} & \textbf{wahr} & \textbf{falsch} \\ \hline
    Beim horizontalen Prototyp werden ausgew"ahlte Teile des Systems durch alle horizontalen Schichten realisiert. & & \\ \hline
    MDA steht auch f"ur Model Driven Architecture. & & \\ \hline
    Aus Modellen generierter Code erf"ullt die Anforderungen stets ohne weitere Pr"ufung. & & \\ \hline
    Code and Fix ist auch ein Prozessmodell. & & \\ \hline
\end{tabular}

\question[3] \textbf{Erg"anzen} Sie folgende Aussagen des \textbf{Agilen Manifests}.
\addpoints
\begin{parts}
    \part \underline{\hspace{5cm}} sind wichtiger als Prozesse und Werkzeuge.
    \part \underline{\hspace{5cm}} sind wichter als ausf"uhrliche Dokumentation.
    \part \underline{\hspace{5cm}} ist wichtiger, als einem Plan zu folgen.
\end{parts}

\question[2] Erg"anzen Sie zwei der folgenden drei S"atze aus dem \textbf{Agilen Manifest von 2001}.
\addpoints
\begin{parts}
    \part Individuen und Interaktionen sind wichtiger als \underline{\hspace{5cm}}.
    \part \underline{\hspace{5cm}} sind wichtiger als ausf"uhrliche Dokumentation.
    \part Die Zusammenarbeit mit dem Kunden ist wichtiger als \underline{\hspace{5cm}}.
\end{parts}

\question[2] Erkl"aren Sie kurz den Unterschied zwischen \textbf{explorativem} und \textbf{evolution"arem Prototyping}.
\addpoints

\question[4] Was ist der \textbf{Unterschied} zwischen \textbf{explorativem} und \textbf{evolution"arem Prototyping}? Wann wird welcher Typ eingesetzt?
\addpoints

{
\checkboxchar{ }
\question \textbf{Nennen} Sie \textbf{zwei Vorteile} und \textbf{zwei Nachteile} des \textbf{Prototyping}.
\addpoints
\begin{checkboxes}
    \choice \textbf{+}
    \choice \textbf{+}
    \choice \textbf{-}
    \choice \textbf{-}
\end{checkboxes}
}

\question[4] Erkl"aren Sie die folgenden Begriffe und nennen sie jeweils ein Beispiel.
\addpoints
\begin{parts}
    \part Wegwerf-Prototyp
    \part Evolution"arer Prototyp
    \part Horizontaler Prototyp
    \part Vertikaler Prototyp
\end{parts}

\question[3] Erg"anzen Sie die fehlenden Phasen des V-Modells in der Abbildung. \\
\addpoints
\begin{tikzpicture}[rectangle/.style={draw = black, fill = white, minimum width = 3cm, minimum height = 1cm}]
    \draw (0, 0) -- (5, -10) -- (10, 0);
    \foreach \xshift/\yshift in {0, 1/2, 3/6, 4/8, 5/10, 8/4}
        \node[xshift = \xshift cm, yshift = -\yshift cm, rectangle] {};
    \node[xshift = 2cm, yshift = -4cm, rectangle] {SW-/HW-Anforderungsanalyse};
    \node[xshift = 10cm, rectangle] {"Uberleitung in die Nutzung};
    \node[xshift = 9cm, yshift = -2cm, rectangle] {System-Integration};
\end{tikzpicture}

\question[4] Nennen sie zwei gute und zwei schlechte Kriterien f"ur das Ende von Pr"ufaktivit"aten.
\addpoints

\question[5] In zwei unabh"angigen Pr"ufvorg"angen des selben St"uck Software wurden einmal 100 Fehler und einmal 60 Fehler gefunden. Davon wurden 30 Fehler in beiden Pr"ufvorg"angen gefunden.
\noaddpoints
\begin{parts}
    \part[1] Wie nennt man das Verfahren, mit dem aus den oben gegebenen Informationen die Gesamtfehlerzahl gesch"atzt werden kann?
    \part[2] Wie hoch wird die Gesamtfehlerzahl f"ur das gegebene Beispiel mit dieser Methode gesch"atzt?
    \part[2] Unter welchen Umst"anden sollte diese Methode besser nicht angewendet werden? Nennen sie zwei F"alle.
\end{parts}

\question[2] Bei einem Review mit zwei Gutachtern findet der eine Gutachter 147 Fehler, der andere 122 davon verschiedene Fehler. Enth"alt die Software vermutlich weitere Fehler? Begr"unden Sie Ihre Antwort.
\addpoints

\question[5] Von einem St"uck Software wissen wir, dass es 20 echte Fehler enth"alt. Nach der Durchf"uhrung einer Restfehlersch"atzung mittels error seeding haben die Pr"ufer von den 15 injizierten Fehlern drei gefunden. Wie viele echte Fehler wurden zur Restfehlersch"atzung injiziert? Welche Schw"ache hat dieses Verfahren der Restfehlersch"atzung?
\addpoints

\question[5] Von einem St"uck Software wissen wir, dass es 40 echte Fehler enth"alt. Nach der Durchf"uhrung einer Restfehlersch"atzung mittels error seeding haben die Pr"ufer acht echte Fehler und f"unf injizierte Fehler gefunden. Wie viele Fehler wurden zur Restfehlersch"atzung injiziert? Welche Schw"ache hat dieses Verfahren der Restfehlersch"atzung?
\addpoints

\question[3] Erkl"aren Sie die Idee der Restfehlersch"atzung durch Error Seeding und leiten Sie die entsprechende Berechnungsformel ab.
\addpoints

\question[4] Erkl"aren Sie die zugrundeliegende Idee der Defensiven Programmierung und Redundanten Programmierung. Erl"autern Sie auch die entsprechenden Umsetzungsm"oglichkeiten.
\addpoints

\question[6] Geben sie durch Ankreuzen eines der Antwortfelder an, ob die folgenden Aussagen wahr oder falsch sind. \\
\addpoints
\begin{tabular}{| p{12cm} | c | c |} \hline
    Aussage & wahr & falsch \\ \hline
    Eine dynamische Pr"ufung kann im V-Modell zu jedem Zeitpunkt durchgef"uhrt werden. & & \\ \hline
    Der letzte Schritt eines Reviews ist stets die Nacharbeit. & & \\ \hline
    Der Vorteil von Reviews ist, dass sie bei hoher Wirksamkeit nur wenig Kosten verursachen. & & \\ \hline
    Je kleiner das Vorhaben ist, desto gr"o"ser sollte der Anteil der Arbeitsergebnisse sein, der durch Reviews gepr"uft wird. & & \\ \hline
    Die Erhebung von Metriken ist kein systematischer Test. & & \\ \hline
    Ein Audit kann ein Review ersetzen. & & \\ \hline
\end{tabular}

\question[6] Geben sie durch Ankreuzen eines der Antwortfelder an, ob die folgenden Aussagen wahr oder falsch sind. \\
\addpoints
\begin{tabular}{| p{12cm} | c | c |} \hline
    Aussage & wahr & falsch \\ \hline
    Der Glasbox-Test eignet sich gut zum Testen ganzer Systeme. & & \\ \hline
    Im Schnitt sieben Fehler pro 1000 Programmzeilen sind normal f"ur sicherheitskritische Software. & & \\ \hline
    Testen ist eine typische konstruktive Qualit"atssicherungsma"snahme. & & \\ \hline
    Beim systematischen Testen geht es darum, die Korrektheit der Software nachzuweisen. & & \\ \hline
    An Dokumenten f"ur ein technisches Review braucht man neben dem Pr"ufling nur noch Referenzunterlagen (zum Beispiel Spezifikation, Richtlinien, Fragenkataloge, Standards). & & \\ \hline
    Der Einsatz eines Vorgehensmodells ist eine analytische Qualit"atssicherungsma"snahme. & & \\ \hline
\end{tabular}

\question[8] Geben sie durch Ankreuzen eines der Antwortfelder an, ob die folgenden Aussagen wahr oder falsch sind. \\
\addpoints
\begin{tabular}{| p{12cm} | c | c |} \hline
    Aussage & Wahr & Falsch \\ \hline
    Ein Systemtest testet das gesamte Softwaresystem in der realen Umgebung. & & \\ \hline
    Zwei Testf"alle sind hinsichtlich des testerfolgs stark "aquivalent, wenn beide geeignet sind, einen bestimmten Fehler anzuzeigen. & & \\ \hline
    Die Pfad"uberdeckung ist zu 100\% erf"ullt, sobald die Zweig"uberdeckung zu 100\% erf"ullt ist. & & \\ \hline
    Mutationstests sind strukturelle Tests. & & \\ \hline
    Der Zweck einer statischen Pr"ufung ist f"u die Pr"ufung irrelevant. & & \\ \hline
    Der letzte Schritt eines Reviews ist stets die Nacharbeit. & & \\ \hline
    Bei einer Restfehlersch"atzung mittels error seeding wurden 25 Fehler injeziert. Die Pr"ufer haben 8 echte Fehler und 5 injuzierte Fehler gefunden. Daraus l"asst sich schlie"sen, dass 40 echte Fehler existieren. & & \\ \hline
    Die Erhebung von Metriken ist ein systematischer Test. & & \\ \hline1
\end{tabular}

\question[8] Geben sie durch Ankreuzen eines der antwortfelder an, ob die folgenden Aussagen wahr oder falsch sind. \\
\addpoints
\begin{tabular}{| p{12cm} | c | c |} \hline
    Aussage & Wahr & Falsch \\ \hline
    "Aquivalenzklassen beim Blackbox-Test sind Teilbereiche des eingabebereichs einer funktion, die sich bez"uglich ihres Funktionswerts gleich verhalten. & & \\ \hline
    Die Zyklomatische Komplexit"at basiert auf der Annahme, dass die Komplexit"at eines Programms von der Anzahl der Anweisungen abh"angt. & & \\ \hline
    Die L"ange des Benutzerhandbuchs ist ein Softwareattribut. & & \\ \hline
    Es ist generell m"oglich, bei einem Glassbox-Test immer alle vorhandenen Pfade auszuf"uhren. & & \\ \hline
    Ein guter Test beginnt mit der Planungsphase. & & \\ \hline
    Unter Plausibilit"at einer Metrik versteht man, dass die empirische Bewertung der Ma"szahl mit den Beobachtungen korelliert. & & \\ \hline
    Inspektion ist eine Form des Reviews. & & \\ \hline
    Bei einem Review wird neben dem Pr"ufling auch der Autor begutachtet. & & \\ \hline
\end{tabular}

\question[5] Nennen sie f"ur die Phasen Anforerungsanalyse und Entwurf
\noaddpoints
\begin{parts}
    \part[2] den Zusammenhand beider Phasen.
    \part[1] eine Gemeinsamkeit beider Phasen.
    \part[2] zwei Unterschiede beider Phasen.
\end{parts}
\addpoints

\question[2] Nennen sie zwei Aspekte, die beim entwurd verglichen mit der Analyse zus"atzlich ber"ucksichtigt werden m"ussen.
\addpoints

\question[4] Nennen sie vier Entwurdskonzepte.
\addpoints

\question[6] Modularit"at ist eines der wichtigsten Entwurfskonzepte. Nennen und erl"autern sie drei weitere wichtige Entwurfskonzepte.
\addpoints

\question[3] Nennen sie je ein Beispiel f"ur funktionale Abstraktion, Datenabstraktion und Kotrollabstraktion.
\addpoints

\question[2] Nennen sie zwei positive Effekte durch eine hohe Modularit"at beim Software-Entwurf?
\addpoints

\question[4] Ein guter objektorientierter Entwurd hat einen starken logischen Zusammenhang innerhalb einer Komponente und pr"azise definierte, minimale Schnittstellen zwischen den Komponenten. Wie hei"sen die beiden zugeh"origen Qualit"atskriterien? Geben sie die Formel an, die aus diesen beiden Kriterien das Optimum berechnet.
\addpoints

\question[2] Erkl"aren sie kurz die Begriffe Koh"asion und Kopplung.
\addpoints

\question[4] Welcher Zusammenhang besteht zwischen Koh"asion und Kopplung und der Qualit"at eines Software-Entwurfs?
\addpoints

\question[8] Vervollst"andigen Sie das UML-Klassendiagramm in der Abbildung so, dass es folgenden Sachverhalt als Dom"anenmodell beschreibt. Vergessen Sie nicht, auch die Kardinalit"aten anzugeben! \\
\emph{Tutorien an der Universit"at Ulm \\Ein Tutor hat einen Namen, eine E-Mail-Adresse und eine Adresse. Er oder die h"alt mindestens ein Tutorium. Kedes Tutorium wird von genau einem tutor geleitet. Ein Student (mit Name, Matrikelnummer und E-Mail) nimmt an genau einem Tutorium teil. An einem Tutorium k"onnen maximal 20 Studenten teilnehmen. Es gibt keine Tutorien ohne Teilnehmer. Tutorien finden an einem oder mehreren Terminen (mit Datum, Uhrzeit und Dauer) statt. Da mehrere R"aume zur Verf"ugung stehen, k"onnen Tutorien auch parallel stattfinden. Ein Stundenplan besteht aus beliebig vielen Terminen. Ein Termin kann auch in mehreren Stundenpl"anen auftauchen, muss aber in mindestens einem vorhanden sein. Auch leere Stundenpl"ane sind m"oglich. Einem Termin ist ein Raum zugeordnet. R"aume haben eine Raumbezeichung.} \\
\addpoints
\begin{tikzpicture}[title/.style={rectangle, minimum height = 2cm, minimum width = 3cm, draw = black}]
    \matrix[column sep = 1.5cm, row sep = 1cm]{
        %1st row
        &
        &
        \node[title] (02) {}; \\
        %2nd row
        \node[title] (10) {}; &
        \node[title] (11) {}; &
        \node[title] (12) {}; \\
        %3rd row
        &
        \node[title] (21) {}; &
        \node[title] (22) {}; \\
    };
    
    \draw (02) -- (12);
    \draw (10) -- (11);
    \draw (11) -- (12);
    \draw (11) -- (21);
    \draw (12) -- (22);
\end{tikzpicture}

\question[5] F"ur ein System sind folgende Anforderungen gegeben:
\begin{itemize}
    \item \emph{Das System verwaltet Pl"atze, Reservierungen, S"ale und Vorstellungen.}
    \item \emph{Eine Reservierung besteht aus mehreren Pl"atzen (mindestens einem) und geh"ort zu genau einer Vorstellung.}
    \item \emph{Zu einer vorstellung kann es mehrere (aber auch keine) Reservierung/-en geben.}
    \item \emph{Eine Vorstellung findet in genau einem Saal statt. Ein Platz geh"ort zu einem Saal. Ein Saal besteht aus mindestens 10 Pl"atzen.}
    \item \emph{In einem Saal k"onnen beliebig viele Vorstellungen stattfinden.}
\end{itemize}
Erg"anzen Sie das Klassendiagramm in der Abbildung um die Kardinalit"aten, sodass die Anforderungen erf"ullt sind. Fehlende Angaben m"ussen Sie sinnvoll erg"anzen. Welches Problem entsteht und wie k"onnte man es l"osen? \\
\addpoints
\begin{tikzpicture}[rectangle/.style = {minimum height = 1.5cm, minimum width = 3cm, draw = black}]
    \matrix[column sep = 3cm, row sep = 3cm]{
        %1st row
        \node[rectangle] (00) {}; &
        \node[rectangle] (01) {}; \\
        %2nd row
        \node[rectangle] (10) {}; &
        \node[rectangle] (11) {}; \\
    };
    \draw (00) -- (01);
    \draw (00) -- (10);
    \draw (01) -- (11);
    \draw (10) -- (11);
\end{tikzpicture}

\question[5] Nennen Sie f"ur die Phasen Anforderungsanalyse und Entwurf
\noaddpoints
\begin{parts}
    \part[2] den Zusammenhang beider Phasen.
    \part[1] eine Gemeinsamkeit beider Phasen.
    \part[2] zwei Unterschiede beider Phasen.
\end{parts}

\question[2] Nennen sie zwei Aspekte, die beim Entwurf verglichen mit der Analyse zus"atzlich ber"ucksichtigt werden m"ussen.
\addpoints

\question[4] Nennen sie vier Entwurfskonzepte.
\addpoints

\question[6] Modularit"at ist eines der wichtigsten Entwurfskonzepte. Nennen und erl"autern sie drei weitere wichtige Entwurfskonzepte.
\addpoints

\question[3] Nennen sie je ein Beispiel f"ur funktionale Abstraktion, Datenbankabstraktion und Kontrollabstraktion.
\addpoints

\question[2] Nennen sie zwei positive Effekte durch hohe Modularit"at beim Software-Entwurf.
\addpoints

\question[4] Ein guter objektorientierter Entwurd hat einen starken logischen Zusammenhang innerhalb einer Komponente und pr"azise definierte, minimale Schnittstellen zwischen den Komponenten. Wie hei"sen die beiden zugeh"origen Qualit"atskriterien? Geben sie die Formel an, die aus diesen beiden Kriterien das Optimum berechnet.
\addpoints

\question[2] Erkl"aren sie kurz die Begriffe Koh"asion und Kopplung.
\addpoints

\question[4] Welcher Zusammenhang besteht zwischen Koh"asion und Kopplung und der Qualit"at eines Software-Entwurfs?
\addpoints

\question[2] Die Abbildung zeigt zwei Entwurfsskizzen. Die Rechtecke symbolisieren die Module, die Verbindungen die Kommunikationswege. Welche der beiden Entwurfsskizzen ist unter dem Aspekt der Kopplung besser zu bewerten? Begr"unden sie ihre Antwort kurz.\\
\addpoints
\begin{tikzpicture}[scale = .5]
    \begin{scope}[rectangle/.style={draw = black, text width = 1cm, minimum height = 5mm}]
        \node (0) at (2, 0) [rectangle] {};
        \node (1) at (6, 0) [rectangle] {};
        \node (2) at (8, 4) [rectangle] {};
        \node (3) at (4, 7) [rectangle] {};
        \node (4) at (0, 4) [rectangle] {};

        \draw (0) -- (1);
        \draw (0) -- (3);
        \draw (1) -- (2);
        \draw (1) -- (3);
        \draw (2) -- (3);
        \draw (2) -- (4);
        \draw (3) -- (4);
        \draw (4) -- (0);
    \end{scope}

    \begin{scope}[xshift = 14cm, rectangle/.style={draw = black, text width = 1cm, minimum height = 5mm}]
        \node (0) at (2, 0) [rectangle] {};
        \node (1) at (6, 0) [rectangle] {};
        \node (2) at (8, 4) [rectangle] {};
        \node (3) at (4, 7) [rectangle] {};
        \node (4) at (0, 4) [rectangle] {};
        \node (5) at (4, 4) [rectangle] {};

        \draw (0) -- (5);
        \draw (1) -- (5);
        \draw (2) -- (5);
        \draw (3) -- (5);
        \draw (4) -- (5);
    \end{scope}
\end{tikzpicture}

\question[4] Sind folgende Aussagen wahr oder falsch? Begr"unden sie ihre Antwort.
\addpoints
\begin{parts}
    \part Auf den Feinentwurf kann man verzichten, auf den Implementierungsentwurf jedoch nicht.
    \part Aufgrund der Struktur in einem DFD kann auf ein Transaktionszentrum geschlossen werden.
    \part Eine gute Modularisierung ist eine wichtige Voraussetzung.
    \part Ein Vorteil des Repositorymodells ist die Effizienz f"ur gro"se Datenbest"ande.
\end{parts}

\question[6] Nennen und beschreiben sie zwei der aus der Vorlesung bekannten Systemstrukturen. Nennen sie zu jedem jeweils zwei Vor- und Nachteile.
\addpoints

\question[4] Erkl"aren sie die Steuerstruktur 'Zentralistische Steuerung' anhand zweier m"oglicher Auspr"agungen.
\addpoints

\question[2] Nennen sie je eine m"ogliche Auspr"agung f"ur die Kontrollmodelle 'Zentralistische Steuerung' und 'Ereignisbasierte Steuerung'.
\addpoints

\question[4] Die modulare Zerlegung beschreibt die Zerlegung von Subsystemen in Module. Nennen sie zwei M"oglichkeiten der modularen Zerlegung und erl"autern sie diese kurz.
\addpoints

\question[4] Erkl"aren sie den Unterschied zwischen ereignisbasierter und zeitbasierter Steuerung im Rahmen des Architekturenentwurfs. In welchem Kontext sollte welches der beiden Kontrollmodelle eingesetzt werden?
\addpoints

\question[4] Nennen sie jeweils zwei Vor- und Nachteile des Client-Server-Modells.
\addpoints

\question[6] Geben sie durch Ankreuzen eines der Antwortfelder an, ob die folgenden Aussagen wahr oder falsch sind. \\
\addpoints
\begin{tabular}{| p{12cm} | l | l |} \hline
    Aussage & Wahr & Falsch \\ \hline
    Ein Vorteil des Repositorymodells ist die Effizient f"ur gro"se Datenbest"ande. & & \\ \hline
    Abstract-Machine-Modell wird auch "'Schichtenmodell"` genannt. & & \\ \hline
    Der strukturierte Entwurf verwendet das Prinzip des Bottom-Up-Vorgehens. & & \\ \hline
    Funktionsorientierte Entw"urfe eignen sich besonders f"ur Systeme mit komplexen Datenstrukturen. & & \\ \hline
    Auf den Feinentwurf kann man verzichten, auf den Implementierungsentwurf nicht. & & \\ \hline
    Eine gute Modularisierung ist eine wichtige Voraussetzung f"ur eine arbeitsteilige Implementierung. & & \\ \hline
\end{tabular}

\question[3] Sie bekommen den Auftrag, die Benutzerschnittstelle f"ur ein neues System zu entwerfen.
\noaddpoints
\begin{parts}
    \part[1.5] Welche drei grunds"atzlichen Aspekte m"ussen sie dabei ber"ucksichtigen?
    \part[1.5] Nennen sie drei Punkte, die sie kl"aren m"ussen, um eine gute Benutzerschnittstelle zu entwerfen.
\end{parts}

\question[3] Geben sie durch Ankreuzen eines der Antwortfelder an, ob die folgenden Aussagen wahr oder falsch sind. \\
\addpoints
\begin{tabular}{| p{12cm} | l | l |} \hline
    Aussage & Wahr & Falsch \\ \hline
    Pair-Programming ist eine Auspr"agung des Ego-less-Programming. & & \\ \hline
    Bei einer guten Kommentierung wird der Programmtext durch Umgangssprache paraphrasiert. & & \\ \hline
    Mit dem Begriff meta CASE werden Werkzeuge bezeichnet, die andere Werkzeuge beinhalten, zusammenfassen oder einfach nur benutzen. & & \\ \hline
\end{tabular}

\question[2] Was ist bei Dokumentation grunds"atzlich zu beachten?
\addpoints

\question[2] Eine Grundvoraussetzung f"ur grunds"atzliche Dokumentation in einem Projekt ist, dass Verantwortlichkeiten und Anforderungen an die Dokumentation klar geregelt sind. Nennen sie vier weitere Grundvoraussetzungen f"ur zus"atzliche Dokumentationen.
\addpoints

\question[2] Nennen sie zwei Aspekte, die grunds"atzlich bei der zus"atzlichen Dokumentation beachtet werden sollen.
\addpoints

\question[3] Welche Aspekte hinsichtlich Werkzeuganforderungen m"ussen bei der Werkzeugwahl ber"ucksichtigt werden? Nennen sie vier Aspekte und erl"autern sie zwei davon anhand von Stichworten.
\addpoints

\question[3] Nennen sie drei Kriterien zur Klassifikation von Werkzeugen.
\addpoints

\question[4] Nennen sie vier Probleme bei der Auswahl von Werkzeugen f"ur die Softwareerstellung.
\addpoints

\question[3] Geben sie durch Ankreuzen eines der Antwortfelder an, ob die folgenden Aussagen wahr oder falsch sind. \\
\addpoints
\begin{tabular}{| p{12cm} | l | l |} \hline
    Aussage & Wahr & Falsch \\ \hline
    Wichtig bei Programmierrichtlinien ist nicht, dass etwas geregelt wird, sondern wie genau etwas geregelt wird. & & \\ \hline
    Rund 90\% aller Programmierer sind mit der Wartung und Verbesserung von altem Code besch"aftigt. & & \\ \hline
    Mit dem Begriff upper CASE werden Werkzeuge bezeichnet, die in den fr"uhen Phasen der Entwicklung eingesetzt werden. & & \\ \hline
\end{tabular}

\question[6] Nennen sie zwei Beispiele f"ur Qualit"atsmerkmale von Software und zwei Beispiele f"ur Softwareattribute und erl"autern sie, wie diese zusammenh"angen.
\addpoints

\question[4] Differenziertheit und Vergleichbarkeit sind zwei wichtige Eigenschaften von Metriken. Nennen sie vier weitere wichtige Eigenschaften von Metriken und erkl"aren sie zwei davon mit jeweils einem Satz.
\addpoints

\question[4] Um die Qualit"at der Codedokumentation zu messen, wird eine Metrik mit $Q = \frac{\text{Anzahl der Codezeilen}}{\text{Anzahl der Kommentarzeilen}}$ definiert, wobei ein h"oherer Wert $Q$ eine h"ohere Qualit"at anzeigt. Zeigen sie, dass diese Metrik nicht plausibel ist.
\addpoints

\question[4] Nennen sie vier wichtige Eigenschaften von Metriken.
\addpoints

\question[3] Nennen sie drei wichtige Eigenschaften von Metriken und erl"autern sie diese jeweils in einem Satz.
\addpoints

\question[2] Was wird unter einer Metrix im Bereich Softwaretechnik verstanden?
\addpoints

\question[2] Die Abbildung zeigt den Ablaufgraph eines Bubblesort-Algorithmus. Geben sie die zyklomatische Komplexit"at des Algorithmus an und erkl"aren sie kurz, wie sie zu dem Ergebnis gekommen sind. \\
\addpoints
\begin{tikzpicture}[circle/.style={draw = black, minimum size = 1cm}]
    \node[circle] (3) {$3$};
    \node[circle] (4) [below = of 3] {$4$};
    \node[circle] (6) [below = of 4,
                        label = right:do] {$6$};
    \node[circle] (7) [below = of 6] {$7$};

    \node[circle] (15) [below left = of 7] {$15$};
    \node[circle] (16) [below = of 15] {$16$};
    \node[circle] (blankleft) [below = of 16] {};

    \node[circle] (9) [below right = of 7,
                        label = right:if] {$9$};
    \node[circle] (1112) [below = of 9] {$11, 12$};
    \node[circle] (blankright) [below = of 1112] {};

    \draw[->] (3) to (4);
    \draw[->] (4) to (6);
    \draw[->] (6) to (7);
    
    \draw[->] (7) to
        node[fill = white] {else}
        (15);
    \draw[->] (15) to (16);
    \draw[->] (16) to (blankleft);
    \draw[->, bend left = 45] (16) to (4);
    
    \draw[->] (7) to
        node[fill = white] {$i < n-1$}
        (9);
    \draw[->] (9) to 
        node[fill = white] {$A[i] > A[i+1]$}
        (1112);
    \draw[->, bend right = 75] (9) to
        node[fill = white] {else}
        (blankright);
    \draw[->] (1112) to (blankright);
    \draw[->, bend right = 90] (blankright) to (7);
\end{tikzpicture}

\question[3] Welche drei Arten von Qualit"atssicherungsma"snahmen gibt es? Nennen sie heweils die Ma"snahmenklasse und erkl"aren sie diese in jeweils einem Satz.
\addpoints

\question[4] Die zyklomatische Komplexit"at (McCabe-Metrik) misst die Komplexit"at eines Programms.
\noaddpoints
\begin{parts}
    \part[3] Zeichnen sie den Ablaufgraphen zu
        \begin{lstlisting}
            do {
                A;
                if B then C; else D;
            } while E;
        \end{lstlisting}    
    \part[1] Berechnen sie anhand des oben erzeugten Ablaufgraphen die zyklomatische Komplexit"at.
\end{parts}
\addpoints

\question[7] Erl"autern sie den grunds"atzlichen Ablauf eines Tests, indem sie die Abbildung vervollst"andigen. Geben sie auch die jeweiligen Ergebnisse der Phasen an. \\
\addpoints
\begin{tikzpicture}[circle/.style={minimum size = 5mm, draw = black},
                    rectangle/.style={draw = black, fill = black!20, minimum height = 8mm, text width = 3cm}]
    \matrix[row sep = 5mm]{
        %1st row
        &
        \node (erg0) {Ergebnis: \underline{\hspace{3cm}}}; &
        & \\
        %2nd row
        \node[rectangle] (rect0) {}; &
        \node[circle] (circ0) {}; &
        & \\
        %3rd row
        &
        \node[rectangle] (rect1) {}; &
        & \\
        %4th row
        &
        &
        \node (erg1) {Ergebnis: \underline{\hspace{3cm}}}; & \\
        %5th row
        &
        \node[rectangle] (rect2) {}; &
        & \\
        %6th row
        &
        &
        \node (erg2) {Ergebnis: \underline{\hspace{3cm}}}; & \\
        %7th row
        &
        \node[rectangle] (rect3) {}; &
        & \\
        %8th row
        &
        \node[circle] (circ1) {}; &
        \node[rectangle] (rect4) {}; & \\
        %9th row
        \node (erg3) {Ergebnis: \underline{\hspace{3cm}}}; &
        &
        & \\
    };

    \draw (erg0) to (rect0.east);
    \draw[->] (rect0) to (circ0);
    \draw[->] (circ0) to (rect1);
    \draw (rect1) to (erg1);
    \draw[->] (rect1) to (rect2);
    \draw (rect2) to (erg2);
    \draw[->] (rect2) to (rect3);
    \draw (rect3) to (erg3);
    \draw[->] (rect3) to (circ1);
    \draw[->] (circ1) to (rect4);
\end{tikzpicture}

\question[6] Gegeben sei die folgende Funktionsdeklaration: \\\lstinline{function Test(eingabe: integer): integer}\\ F"ur den Wertebereich des Parameters \texttt{eingabe} gilt: $10 \leq \texttt{eingabe} < 100$. Die Funktion \texttt{Test} soll durch ein Blackbox-Verfahren getestet werden.
\addpoints
\begin{parts}
    \part Was sind sinnvolle "Aquivalenzklassen f"ur diese Funktion?
    \part Geben sie f"ur jede "Aquivalenzklasse jeweils einen geeigneten Testfall an.
\end{parts}

\question[4] Nennen sie zwei Arten von Fehlern, welche mit Testen nicht gefunden werden k"onnen. Welche Alternativen stehen zur Verf"ugung, um diese Arten von Fehlern zu finden?
\addpoints

\question[2] Nennen sie zwei in der Vorlesung behandelte "Uberdeckungskriterien.
\addpoints

\question[2] Ist es generell bei einem Glassbox-Test immer m"oglich, dass jeder m"ogliche Pfas in einer Programmeinheit ausgef"uhrt werden kann? Begr"unden sie kurz ihre Meinung.
\addpoints

\question[12] Betrachten sie folgendes Listing.
\noaddpoints
\begin{lstlisting}
    void Bubblesort( List<int> a )
    {
        int n = A.Count();
        do
        {
            bool vertauscht = false;
            for (int i=0; i < n-1; i++)
            {
                if (A[i] > A[i+1])
                {
                    vertausche (A, i, i+1);
                    vertauscht = true;
                }
            }
            n = n - 1;
        } while(vertauscht && n>1);
    }
\end{lstlisting}
\begin{parts}
    \part[5] Die Methode soll einem Blackbox-Test unterzogen werden. Definieren sie f"unf sinnvolle Testf"alle. Achten sie auf eine m"oglichst gute "Uberdeckung.
    \part[2] Der Bubblesort-Algorithmus aus dem Listing wird aus Performancegr"unden durch den Quicksort-Algorithmus ersetzt. Definieren sie wieder f"unf sinnvolle Testf"alle f"ur einen Blackbox-Test ud begr"unden sie kurz ihre Wahl. Sie d"urfen sich bei ihrer Antwort auf Teilaufgabe a) beziehen.
    \part[5] Als Testfall eines Glassbox-Tests wird die Eingabe A = ${1, 2, 5, 4}$ angenommen. Welche Arten der "Uberdeckung erreicht man durch diesen Testfall? Begr"unden sie ihre Antwort kurz.
\end{parts}

\question[6] Gegeben seien ein Listing und der zugeh"orige Ablaufgraph f"ur einen Glassbox-Test. Als Testfall f"ur die Funktion im Listing wird die Konfiguration $k = 3, s = [1, 2, 3, 4, 5, 6, 7, 8]$ angenommen. Was f"ur ein "Uberdeckungsgrad wird durch diese Konfiguration erreicht? Begr"unden sie ihre Antwort kurz.
\addpoints
\begin{lstlisting}
    function linsearch (k: integer, s: intarray): nat
        var bot: nat = 1, top: nat = len(s), mid: nat,
            found: bool = false, i: nat = 0

        begin
        while bot <= top do
            if found
                then exit
                else mid = (top + bot) div 2
                    if s[mid] < k
                        then found = true, i = mid
                        elsif s[mid] < k
                            then bot = mid + 1
                            else top = mid - 1
                    fi
            fi od
        return i
\end{lstlisting}
\begin{tikzpicture}[circle/.style={draw = black, minimum size = 1cm}]
    \node[circle] (1) {1};
    \node[circle] (2) [below = of 1] {2};
    \node[circle] (3) [below = of 2, label = right:While bot <{=} top] {3};
    \node[circle] (4) [below left = of 3, label = right:Found 7] {4};
    \node[circle] (5) [below right = of 3, label = right:s{[mid]} {=} k?] {5};
    \node[circle] (6) [below left = of 5] {6};
    \node[circle] (7) [below right = of 5, label = right:s{[mid]} < k?] {7};
    \node[circle] (8) [below left = of 7] {8};
    \node[circle] (9) [below right = of 7] {9};
    \node[circle] (10) [below right = of 8] {10};
    \node[circle] (11) [below left = of 10] {11};
    \node[circle] (12) at (4 |- 11) {12};
    \node[circle] (13) [below = of 12] {13};

    \draw[->] (1) to (2);
    \draw[->] (2) to (3);
    \draw[->, bend right = 45] (2) to (12);
    \draw[->, bend right = 45] (3) to (4);
    \draw[->, bend left = 45] (3) to (5);
    \draw[->, bend right = 45] (5) to (6);
    \draw[->, bend left = 45] (5) to (7);
    \draw[->, bend right = 22.5] (6) to (11);
    \draw[->, bend right = 45] (7) to (8);
    \draw[->, bend left = 45] (7) to (9);
    \draw[->, bend right = 45] (8) to (10);
    \draw[->, bend left = 45] (9) to (10);
    \draw[->, bend right = 45] (10) to (11);
    \draw[->, bend right = 90] (11) to (2);
    \draw[->] (12) to (13);
\end{tikzpicture}

\question[3] Nennen sie drei wesentliche T"atigkeiten, die als Teil der Testvorbereitung durchgef"uhrt werden m"ussen.
\addpoints

\question[4] Nennen sie zwei St"arken und zwei Schw"achen von Tests.
\addpoints

\question[5] Geben sie an, ob folgende Aussagen wahr oder falsch sind. Begr"unden sie ihre Antwort jeweils.
\addpoints
\begin{parts}
    \part "Aquivalenzklassen bei Black-Box-Test sind Teilbereich des Eingabebereichs einer Funktion die sich bez"uglich ihres Funktionswerts gleich verhalten.
    \part Es ist generell m"oglich, bei einem Glassbox-Test immer alle vorhandenen Pfade auszuf"uhren.
    \part Eine 100\%ige Zweig"uberdeckung impliziert eine 100\%ige Anweisungs"uberdeckung.
    \part Bei einer Restfehlersch"atzung mittels Error Seeding wurden 25 Fehler injiziert. Die Pr"ufer haben acht echte Fehler und f"unf injizierte Fehler gefunden. Daraus l"asst sich schlie"sen, dass 40 echte Fehler existieren.
    \part Eine dynamische Pr"ufung kann im V-Modell zu jedem Zeitpunkt durchgef"uhrt werden.
\end{parts}

\question[2] Erkl"aren sie, warum Laufversuche nicht als Test in der Qualit"atssicherung gelten k"onnen.
\addpoints

\question[2] Zwei wichtige Teststrategien sind Black Box- und White Box-Tests.
\noaddpoints
\begin{parts}
    \part Nennen sie eine Grundlage, die zur Erstellung von Black Box-Tests notwendig ist.
    \part Nennen sie eine Grundlage, die zur Erstellung von White Box-Tests notwendig ist.
\end{parts}

\question[3] Performanztest ist eine Form des Systemtests. Nennen sie drei weitere Formen des Systemtests und erkl"aren sie jede davon in einem Satz.
\addpoints

\question[3] Die Methode $sqrt(x): double$ berechnet die mathematische Funktion $\sqrt{x}$. Definieren sie eine m"oglichst kleine Menge an Testeingaben f"ur $x$, sodass alle "Aquivalenzklassen und Grenzwerte abgedeckt sind.
\addpoints

\question[6] Erl"autern sie in ganzen S"atzen anhand der auszuf"uhrenden Schritte den prinzipiellen Ablauf eines Reviews.
\addpoints

\question[6] Sie m"ussen ein Review f"ur ein 50 Seiten starkes Dokument planen. F"ur dieses Review ist eine "'Dritte Stunde"` nicht vorgesehen. \\Ein Gutachter schatt 10 Seiten pro Stunde, 3 Gutachter werden ben"otigt. F"ur jeden Teilnehmer des Reviews werden Kosten von 200 Euro pro Stunde veranschlagt. F"ur ihren Zeitaufwand zur Planung k"onnen sie 2 Stunden veranschlagen. \\Berechnen Sie den Gesamtaufwand und die Kosten f"ur Ihr komplettes Review unter Angabe der einzelnen Rechenschritte.
\addpoints

\question[4] Ist ein Review in allen Phasen der Systementwicklung nach dem V-Modell
\begin{parts}
    \part m"oglich?
    \part sinnvoll?
\end{parts}
Begr"unden Sie jeweils ihre Meinung.

\question[5] Nennen Sie f"unf wesentliche Schritte eines Review und die jeweils darin involvierten Rollen.
\addpoints

\question[4] Nennen sie zwei St"arken und zwei Schw"achen von Reviews.
\addpoints

\question[2] Nennen sie vier charakteristische Merkmale f"ur ein Projekt.
\addpoints

\question[4] Nennen sie zwei prim"are und zwei sekund"are Ziele des Projektmanagements.
\addpoints

\question[6] Erl"autern sie die Function-Point-Methode anhand der Ausgangslage und der Vorgehensschritte.
\addpoints

\question[5] Die Function-Point-Methode ist eine Technik zur Aufwandssch"atzung. Erkl"aren sie diese Methode anhand der f"unf Schritte, die durchgef"uhrt werden m"ussen.
\addpoints

\question[2] Function Points sind ein abstraktes Ma"s f"ur den Aufwand. Wie kommt man zu konkreten Zahlenwerten (wie beispielsweise f"ur die Anzahl der ben"otigten Mitarbeiter oder den Codeumfang)?
\addpoints

\question[6] Die Berechnung bewerteter Function Points geschieht mit der Formel $FP = UFP \cdot VAF$, wobei $VAF = 0,65 \cdot 0,01 \cdot TDI$ ist.
\noaddpoints
\begin{parts}
    \part[4] Erkl"aren sie, wie man grunds"atzlich vorgehen muss, um $UFP$ zu erhalten.
    \part[2] Welche Aussage steckt hinter dem Faktor $TDI$? Erkl"aren sie in ganzen S"atzen.
\end{parts}

\question[4] Die Aufwandssch"atzung nach Function Points kann auch mithilfe sogenannter bewerteter Function Points vorgenommen werden. Diese werden (nach IFPUG 1994) folgenderma"sen bewertet: $FP = UFP \cdot VAF$, wobei $UFP$ die (aus den funktionalen Anforderungen bestimmten) unbewerteten Function Points und $VAF$ den Wertkorrekturfaktor bezeichnen. Der $VAF$ setzt sich folgenderma"sen zusammen: $0,65 + 0,01 \cdot TDI$.
\begin{parts}
    \part[2] Welche Aussage steckt hinter dem Faktor $TDI$?
    \part[2] Geben sie einen konkreten Wertebereich an, innerhalb dessen sie der Wertkorrekturfaktor ($VAF$) bewegt.
\end{parts}

\question[4.5] Der Aufwand wird bei COCOMO mittels $A \cdot KDSI^b$ berechnet. $A$ und $b$ sind dabei von der Projektklasse abh"angig. Nennen Sie die verschiedenen Projektklassen von COCOMO und erkl"aren sie, wie diese charakterisiert sind.
\addpoints

\question[2] Nennen sie zwei Unterschiede zwischen COCOMO und COCOMO II.
\addpoints

\question[3] Eine M"oglichkeit, Aufwand zu sch"atzen, ist die Verwendung der Formel $\text{Aufwand} = A \cdot KDSI^b$.
\noaddpoints
\begin{parts}
    \part[1] Um welches Sch"atzverfahren handelt es sich dabei?
    \part[2] F"ur welche Projektcharakteristika stehen die Parameter $A$, $KDSI$, und $b$?
\end{parts}

\question[2] Nennen sie zwei Hauptbeteiligte an einem Projekt.
\addpoints

\question[4.5] Nennen sie die drei wichtigsten Projektbeteiligten/Rollen udn beschreiben sie stichpunktartig deren Aufgaben und Verantwortlichen im Projekt.
\addpoints

\question[4] Brooks' Law besagt: \emph{Adding mapower to a late project makes it even later}. K"onnen sie dieser Aussage zustimmen? Begr"unden sie ihre Meinung. Nennen sie zwei weiter nicht zu empfehlende Ma"snahmen.
\addpoints

\question[6] In ihrem Projekt sto"sen sie nach der H"alfte der veranschlagten Zeit auf ein erhebliches Problem. Sie k"onnen das Projekt deshalb nicht mehr wie geplant zu Ende bringen. Welche Ma"snahmen k"onnen sie ergreifen, um das Projekt trotzdem erfolgreich abschlie"sen zu k"onnen? Nennen sie drei Ma"snahmen und erkl"aren sie diese kurz.
\addpoints

\question[4] Bei ihrem Projekt sto"sen sie nach der H"alfte der veranschlagten Zeit auf erhebliche Probleme. Sie k"onnen das Projekt deshalb nicht wie geplant zu Ende bringen. Welche Ma"snahmen k"onnen sie ergreifen? Nennen sie zwei empfehlenswerte Ma"snahmen und zwei nicht zu empfehlende Ma"snahmen.
\addpoints

\question[6] Die Abbildung zeigt die Produktivit"at eines Projektteams "uber der Zeit. Erkl"aren sie die Ph"anomene, die zu beobachten sind, wenn zum Zeitpunkt $B$ neue Mitarbeiter zum Projektteam hinzugef"ugt werden. Welche Erkenntnisse kann man daraus f"ur das Management gewinnen? \\
\addpoints
\begin{tikzpicture}
    % Achsen
    \draw[->] (0, 0) -- node[xshift = 5cm, yshift = -3mm] {Zeit} (12, 0);
        \foreach \x/\xtext in {2/A, 5/B, 9/C}
        \draw (\x, 3mm) -- node[yshift = -6mm] {\xtext} (\x, -3mm);
    \draw[->] (0, 0) -- node[rotate = 90, xshift = 2cm, yshift = 3mm] {Produktivit"at} (0, 7);

    % Dotted line stuff
    \draw[dashed] (0, 5) -- (12, 5);

    % Fancy shaped thick line stuff
    \draw[very thick] (0, 6) -- (5, 6) -- (5, 2) -- (5, 2) arc (180:90:3) -- (12, 5);
\end{tikzpicture}

\question[6] Benennen sie die Organisationsformen der Projektteams in der Abbildung und nennen sie jeweils einen Vor- und Nachteil zu jeder Form. \\
\addpoints
\begin{tikzpicture}[circle/.style={minimum size = 5mm, draw = black}]
    \matrix[column sep = 2mm, row sep = 5mm]{
       %1st row
       &
       &
       \node[circle] (1) {}; &
       &
       & \\

       %2nd row
       \node[circle] (2) {}; &
       &
       &
       &
       \node[circle] (0) {}; & \\

       %3rd row
       &
       &
       &
       &
       & \\

       %4th row
       &
       \node[circle] (3) {}; &
       &
       \node[circle] (4) {}; &
       & \\
    };

    \draw (0) -- (1);
    \draw (0) -- (2);
    \draw (0) -- (3);
    \draw (0) -- (4);
    \draw (1) -- (2);
    \draw (1) -- (3);
    \draw (1) -- (4);
    \draw (2) -- (3);
    \draw (2) -- (4);
    \draw (3) -- (4);

    \matrix[xshift = 5cm, column sep = 5mm, row sep = 5mm]{
        %1st row
        &
        &
        \node[circle] (02) {};
        &
        & \\

        %2nd row
        &
        \node[circle] (11) {}; &
        \node[circle] (12) {}; &
        \node[circle] (13) {}; &
        & \\

        %3rd row
        \node[circle] (20) {}; &
        \node[circle] (21) {}; &
        \node[circle] (22) {}; &
        \node[circle] (23) {}; &
        \node[circle] (24) {}; & \\
    };

    \draw (02) -- (11);
    \draw (02) -- (12);
    \draw (02) -- (13);
    \draw (11) -- (20);
    \draw (11) -- (21);
    \draw (11) -- (22);
    \draw (13) -- (23);
    \draw (13) -- (24);
\end{tikzpicture}

\question[2] Begründen sie stichpunktartig, warum eine demokratische Teamorganisation für große Softwareprojekte mit mehreren Dutzenden Mitarbeitern ungeeignet ist.
\addpoints

\question[2] Nennen sie zwei Möglichkeiten, ein Projekt in eine Firmenorganisation einzubetten.
\addpoints

\question[4] Innerhalb kürzester Zeit soll eine kleine Webseite entstehen, auf der Besucher biologische Sachverhalte wie die Vererbung am Beispiel unterschiedlich farbiger Mäuse nachvollziehen können. Die Webseite soll über ein kreatives Design verfügen, technisch auf dem neuesten Stand sein, und die Qualität der biologischen Inhalte soll von höchstem Niveau sein. \\Da ihr Unternehmen schon lange Software entwickelt, hat sich in ihrem Unternehmen eine hierarchische Teamorganisation etabliert. Durch welche Argumente überzeugen sie ihren Chef, dass für dieses Projekt eine anarchische Organisation besser geeignet ist?
\addpoints

\question[9] Betrachten sie die drei folgende Szenarien und entscheiden sie je Szenario, wie die Organisationsstruktur des jeweiligen Projektteams aussehen sollte. Begründen sie jeweils kurz ihre Entscheidung.
\addpoints

Szenario 1: Innerhalb kürzester Zeit soll eine kleine Webseite entstehen, auf der besucher biologische Sachverhalte wie die Vererbung am Beispiel unterschiedlich farbiger Mäuse nachvollziehen können. Die Webseite soll über ein kreatives Design verfügen, technisch auf dem neuesten Stand sein und die Qualität der biologischen Inhalte soll von höchstem Niveau sein.

Szenario 2: Für einen Avionik-Zulieferer soll innerhalb von zwei Jahren mit einem großen Team ein neues System zur Steuerung des Kabinendrucks in einem Passagierflugzeug entwickelt werden. Dieses System soll in Zukunft erweitert werden und auch in anderen Flugzeugtypen Verwendung finden.

Szenario 3: Eine sehr kleine Softwarefirma, in der alle Mitarbeiter auch Teilhaber sind, möchte ihre internen Prozesse durch eine selbst entwickelte Software unterstützen.

\question[3] Bewerten sie anhand der Trendanalysen in der Abbildung die Qualität der Projektplanung. \\
\addpoints
\begin{tikzpicture}[scale = 2]
    \begin{scope}
        \draw[->] (0, 0) -- (2, 0);
        \draw[->] (0, 0) -- (0, 2);
        \draw[dotted] (0, 0) -- (2, 2);

        \draw[thick] (0, 0.5) -- (0.4, 0.8) -- (0.5, 0.5);
        \draw[thick] (0, 1) -- (0.2, 0.8) -- (0.3, 1) -- (0.5, 1) -- (0.7, 1.2) -- (1, 1);
        \draw[thick] (0, 1.2) -- (0.3, 1.3) -- (0.4, 1.7) -- (0.5, 1.2) -- (0.6, 1.5) -- (0.7, 1.4) -- (1, 1.3) -- (1.1, 1.4) -- (1.2, 1.2);
    \end{scope}

    \begin{scope}[xshift = 3cm]
        \draw[->] (0, 0) -- (2, 0);
        \draw[->] (0, 0) -- (0, 2);
        \draw[dotted] (0, 0) -- (2, 2);

        \draw[thick, yshift = 4.5mm] (0, 0) -- (0.25, 0) -- (0.5, 0.1) -- (1.95, 1.55);
        \draw[thick, yshift = 5.5mm] (0, 0) -- (0.25, 0.1) -- (0.5, 0.25) -- (1.85, 1.45);
        \draw[thick, yshift = 8mm] (0, 0) -- (0.3, 0.1) -- (0.45, 0.05) -- (1.8, 1.2);
    \end{scope}

    \begin{scope}[xshift = 6cm]
        \draw[->] (0, 0) -- (2, 0);
        \draw[->] (0, 0) -- (0, 2);
        \draw[dotted] (0, 0) -- (2, 2);

        \draw[thick] (0, 1.5) -- (0.3, 1.45) -- (1, 1);
        \draw[thick] (0, 1.7) -- (0.3, 1.7) -- (0.65, 1.65) -- (0.75, 1.6) -- (1.2, 1.2);
        \draw[thick] (0, 1.8) -- (0.2, 1.9) -- (0.5, 1.75) -- (0.7, 1.85) -- (1.2, 1.7) -- (1.3, 1.3);
    \end{scope}
\end{tikzpicture}

\question[1] Sie sind Projektleiter des umfangreichen Softwareprojekts \emph{RouPla}. Ihre Teilprojektleiter für die Teilprojekte \emph{RouPla.1}, \emph{RouPla.2}, und \emph{RouPla.3} schicken ihnen im Verlauf des Projekts folgende Planungsdaten: \\
\begin{tabular}{| r | c | c | c | c | c | c |} \hline
    am & 01.11.09 & 15.11.09 & 01.12.09 & 15.12.09 & 01.01.10 & 15.01.10 \\ \hline
    \emph{RouPla.1} Soll-Fertig & 01.01.10 & 20.01.10 & 30.01.10 & 10.02.10 & 15.02.10 & 25.02.10 \\ \hline
    \emph{RouPla.2} Soll-Fertig & 01.02.10 & 15.01.10 & 30.03.10 & 15.01.10 & 15.02.10 & 01.02.10 \\ \hline
    \emph{RouPla.2} Soll-Fertig & 15.04.10 & 01.04.10 & 20.03.10 & 01.03.10 & 15.02.10 & 10.02.10 \\ \hline
\end{tabular}
\noaddpoints
\begin{parts}
    \part[7] Überprüfen sie die Planungsfähigkeiten ihrer Teilprojektleiter. Zeichnen sie auf Basis der Planungsdaten für jedes Teilprojekt in der Abbildung ein Meilensteintrendanalysediagramm einschließlich Beschriftung der Achsen. \\
    \begin{tikzpicture}[scale = 2]
        \foreach \x/\xshift in {1/0, 2/3, 3/6}{
            \begin{scope}[xshift = \xshift cm]
                \foreach \x in {0.0, 0.3, 0.6, 0.9, 1.2, 1.5, 1.8}
                    \draw (\x, -.5mm) -- (\x, .5mm);
                \foreach \y in {0.0, 0.3, 0.6, 0.9, 1.2, 1.5, 1.8}
                    \draw (-.5mm, \y) -- (.5mm, \y);
                \draw[->] (0, 0) -- node[near end, yshift = 4mm, fill = white] {\footnotesize RouPla\x} (2, 0);
                \draw[->] (0, 0) -- (0, 2);
                \draw[dotted] (0, 0) -- (2, 2);
            \end{scope}
        };
    \end{tikzpicture}
    
    \part[4] Ihr Teilprojektleiter \emph{RouPla.3} liefer ihnen nun zum dritten Mal eine ähnliche Planung ab. Wie beurteilen sie die Planungsfähigkeiten ihres Teilprojektleiters \emph{RouPla.3}? Was sagen sie dem Teilprojektleiter? Begründen sie ihre Meinung.
\end{parts}

\question[4] Der Projektverlauf kann mittels Meilensteintrendanalyse verfolgt werden. Zeichnen sie in den Diagrammen in der Abbildung jeweils mindestens eine Trendlinie ein, die zu dem entsprechenden Projekt passen könnte. \\
\begin{tikzpicture}
    \foreach \xshift/\xtext in {0/{Ideales Projekt}, 3.5/{Projekt nicht beherrschbar}, 7/{Projekt ohne Fortschritt}, 10.5/{Projekt mit überschätztem Aufwand}}{
        \begin{scope}[xshift = \xshift cm]
            \draw[->] (0, 0) -- node[below, text width = 2.5cm] {\footnotesize \xtext} (2.5, 0);
            \draw[->] (0, 0) -- (0, 2.5);
            \draw[dotted] (0, 0) -- (2.5, 2.5);
        \end{scope}
    };
\end{tikzpicture}
\addpoints

\question[5] Was sind die vier wesentlichen Tätigkeiten im Risikomanagement? Nennen sie die Tätigkeiten analog zum logischen, zeitlichen Ablauf.
\addpoints

\question[6] Nennen sie die vier Aktivitäten des Risikomanagements und nennen sie zu jeder Aktivität ein mögliches Vorgehen.
\addpoints

\question[3] Nennen sie sechs konkrete Risiken, die den Erfolg von Softwareprojekten gefährden können.
\addpoints

\question[2] Nennen sie je ein Beispiel für ein Produkt- und ein Projektrisiko.
\addpoints

\question[4] Nennen sie zwei Beispiele für Produktrisiken und zwei Beispiele für Projektrisiken.
\addpoints

\question[2] Nennen sie je ein Beispiel für ein Prozess- und ein Projektrisiko.
\addpoints

\question[4] Sie haben in ihrem Projekt ein Risiko identifiziert.
\noaddpoints
\begin{parts}
    \part[2] Nennen sie die Aspekte, welche sie bei der Bewertung des Risikos berücksichtigen müssen.
    \part[2] Nennen sie vier grundsätzliche Möglichkeiten, welche sie zur Verfügung haben, um auf das Risiko zu reagieren.
\end{parts}

\question[10] Eine Risikobewertung kann quantitativ erfolgen. Dazu wird die \emph{Risk Exposure} des Risikos $r$ und sienen möglichen Auswirkungen $a$ mittels $Exp(r) = \sum_aP(a) \cdot S(a)$ berechnet.
\noaddpoints
\begin{parts}
    \part[2] Erklären sie jeweils die Bedeutung von $P(a)$ und $S(a)$.
    \part[2] Auch die Bewertung einer Gegenmaßnahme $m$ kann quantitativ erfolgen. Erklären sie die dazugehórige Formel für die Risk Reduction Leverage $RRL(r, m) = \frac{Exp(r) - Exp (r|m}{K(m)}$. Wie sind die Werte von RRL zu interpretieren?
    \part[6] Wird ein Flugzeug nur alle 20 Flüge gewartet, so kommt es bei 50\% der Flüge zu Komfortverlusten für die Passagiere, bei 25\% der Flüge zu Reiseverzögerungen, und bei 0.001\% der Flüge zu einem Absturz. Ein Komfortverlust kostet die Airline 5000€, eine Reiseverzögerung 50000€ und ein Absturz 100 Mio.€. \\Durch die Halbierung des Wartungsintervalls entstehen 50000€ Zusatzkosten. Dafür treten Reiseverzögerungen nur noch bei 5\% und Abstürze nur noch bei 0,0001\% der Flüge auf. \\Ist die Maßnahme rein finanziell betrachtet sinnvoll? Berechnen sie dazu die RRL der Maßnahme.
\end{parts}
\addpoints

\question[2] Was ist CMM und wofür wird es eingesetzt?

\question[5] Ihr Unternehmen hat bereits Level 2 im Capability Maturity Model (CMM) erreicht und hat demnach grundlegende Kompetenzen in den Schlüsselbereichen Konfigurationsmanagement, Projektplanung, Projekt-Controlling, und Qualitätssicherung. \\Ihr Chef beauftragt sie, innerhalb des nächsten Jahres die CMM-Zertifizierung ihres Unternehmens bis einschließlich Stufe 5 umzusetzen. Da ihm die Zertifizierung besonders wichtig ist, sagt er zu, ihnen sowohl das notwendige Budget als auch notwendiges Personal im von ihnen geforderten Umfang zur Verfügung zu stellen.
\noaddpoints
\begin{parts}
    \part[2] Nennen sie zwei Maßnahmen, die für das Erreichen des Level 5 von Leven 2 aus im CMM mindestens nötig sind.
    \part[3] Lässt sich die ihnen gestellte Aufgabe, innerhalb eines Jahres, von CMM Level 2 auf CMM Level 5 zu kommen, umsetzen? Erläutern sie, welche Maßnahmen notwendig sind, oder begründen sie, warum eine Realisierung des Auftrages nicht möglich ist.
\end{parts}
\addpoints

\begin{comment}
{%
\renewcommand*\choicelabel{\thechoice)}
%
\question[2] Element with $Z=92$ is:
\begin{multicols}{2}
\begin{choices}
\choice H
\choice O
\choice F
\choice S
\choice Ba
\choice Pb
\choice U
\choice Pu
\end{choices}
\end{multicols}
}%

\question[10]
In no more than one paragraph, explain why the earth is round.
\makeemptybox{2in}

\question[20]
Explain blah, blah\ldots
\makeemptybox{\fill}

\newpage

\question[20]
Explain blah, blah\ldots
\fillwithlines{\fill}

\newpage

\question[20]
Explain blah, blah\ldots
\fillwithdottedlines{8em}
\end{comment}

\end{questions}

\end{document}
