\documentclass[12pt]{exam}
\usepackage[utf8]{inputenc}

\usepackage{verbatim}
\usepackage[margin=1in]{geometry}
\usepackage{amsmath,amssymb}
\usepackage{multicol}
\usepackage[ngerman]{babel}
\usepackage{tikz}
\usetikzlibrary{arrows, petri, positioning, shapes}

\usepackage{listings}
\lstset{
    basicstyle = \ttfamily \footnotesize,
    frame = single,
    keepspaces = true,
    numbers = left,
}

\newcommand{\class}{Softwaretechnik}
\newcommand{\term}{Aufgabensammlung}
\newcommand{\examnum}{}
\newcommand{\examdate}{Stand: \today{}}
\newcommand{\timelimit}{Gibts nicht}

\pagestyle{head}
\firstpageheader{}{}{}
\runningheader{\class}{\examnum\ - Page \thepage\ of \numpages}{\examdate}
\runningheadrule


\begin{document}

\noindent
\begin{tabular*}{\textwidth}{l @{\extracolsep{\fill}} r @{\extracolsep{6pt}} l}
\textbf{\class} & \textbf{Name:} & \makebox[2in]{\hrulefill}\\
\textbf{\term} &&\\
\textbf{\examnum} &&\\
\textbf{\examdate} &&\\
\textbf{Time Limit: \timelimit} & Teaching Assistant & \makebox[2in]{\hrulefill}
\end{tabular*}\\
\rule[2ex]{\textwidth}{2pt}

This exam contains \numpages\ pages (including this cover page) and \numquestions\ questions.\\
Total of points is \numpoints.

Rest of introduction. Rest of introduction. Rest of introduction. Rest of introduction. Rest of introduction. Rest of introduction. Rest of introduction. Rest of introduction. 

\begin{comment}
\begin{center}
Grade Table (for teacher use only)\\
\addpoints
\gradetable[v][questions]
\end{center}
\end{comment}

\noindent
\rule[2ex]{\textwidth}{2pt}

\begin{questions}

\question[5] Geben sie durch Ankreuzen an eines der Antwortfelder an, ob die folgenden Aussagen wahr oder falsch sind. \\
\addpoints
\begin{tabular}{| p{12cm} | c | c |} \hline
\textbf{Aussage} & \textbf{Wahr} & \textbf{Falsch} \\ \hline
Auf den Feinentwurf kann man verzichten, auf den Implementierungsentwurf jedoch nicht. & & \\ \hline
Kleine Kohäsion und große Kopplung sind Zeichen eines guten Entwurfs. & & \\ \hline
Eine gute Modularisierung ist eine wichtige Voraussetzung für eine arbeitsteilige Implementierung. & & \\ \hline
Das Transformationszentrum in einem Datenflussdiagramm ergibt sich rein aus dessen Struktur. & & \\ \hline
Ein Verteilungsdiagramm zeigt die Kommunikationsverbindungen und Abhängigkeiten von zur Laufzeit physisch vorhandenen Ressourcen. & & \\ \hline
\end{tabular}

\question[2] Nennen sie \textbf{zwei Vorteile von Entwurfsmustern}.
\addpoints

\question[2] In der unteren Abbildung ist ein schematisches DFD dargestellt. Woran können Sie erkennen, ob es sich bei dem Prozess \textbf{D} um ein Transformationszentrum oder ein Transaktionszentrum handelt? \\
\addpoints
\begin{tikzpicture}
    \matrix[row sep = 1cm, column sep = 1cm]{
        % 1st row
        \node[place, white] (00) {}; &
        \node[place] (A) {A}; &
        \node[place] (B) {B}; &
        \node[place, white] (03) {}; &
        \node[place] (E) {E}; &
        \node[place] (F) {F}; &
        \node[place, white] (06) {}; \\
        % 2nd row
        \node[place, white] (10) {}; &
        \node[place, white] (11) {}; &
        \node[place] (C) {C}; &
        \node[place] (D) {D}; &
        \node[place] (G) {G}; &
        \node[place, white] (15) {}; &
        \node[place, white] (16) {}; \\
        % 3rd row
        \node[place, white] (20) {}; &
        \node[place, white] (21) {}; &
        \node[place, white] (22) {}; &
        \node[place, white] (23) {}; &
        \node[place] (H) {H}; &
        \node[place, white] (25) {}; &
        \node[place, white] (26) {}; \\
    };
    
    \draw[->] (00) -- (A);
    \draw[->] (A) -- (B);
    \draw[->] (B) -- (D);
    \draw[->] (E) -- (F);
    \draw[->] (F) -- (06);
    \draw[->] (11) -- (C);
    \draw[->] (C) -- (D);
    \draw[->] (D) -- (E);
    \draw[->] (D) -- (G);
    \draw[->] (G) -- (15);
    \draw[->] (D) -- (H);
    \draw[->] (H) -- (25);
\end{tikzpicture}

\question[2] \textbf{Erklären} Sie die \textbf{Merkmale der Strukturierten Methoden}.
\addpoints

\question[3] \textbf{Erklären} Sie die \textbf{Bedeutung} folgender Notationselemente aus einem \textbf{Datendlussdiagramm} in jeweils einem Satz.
\addpoints
\begin{parts}
    \part \tikz \draw (0, 0) rectangle (2, 1);
    \part \tikz \draw (0, 0) circle (5mm);
    \part \tikz \draw[->] (0, 0) -- (2, 0);
\end{parts}

\question[2] Wie können bei der \textbf{Strukturierten Analyse} (SA) Prozesse beschrieben werden?

\question[4]
\noaddpoints
\begin{parts}
    \part[2] Erläutern sie kurz den Unterschied zwischen einem \textbf{Programm} und einem \textbf{Softwaresystem}.
    \part[1] \textbf{Kreuzen} Sie \textbf{Ihre Antwort an}. \\Wie hoch ist die durchschnittliche Produktivität pro Person bei sicherheitskritischer Software (inklusive Test und Dokumentation) in \emph{Zeilen ausführbares Programm pro Tag}?
    \begin{checkboxes}
        \choice 50 Zeilen pro Tag
        \choice 25 Zeilen pro Tag
        \choice 15 Zeilen pro Tag
        \choice 2 Zeilen pro Tag
    \end{checkboxes}
    \part[1] \textbf{Kreuzen} Sie {Ihre Antwort an}. \\Wie viel Prozent der Programmierer in Unternehmen sind laut Vorlesung durchschnittlich mit der Wartung und Verbesserung von altem Code beschäftigt?
    \begin{checkboxes}
        \choice $<10\%$
        \choice $10-29\%$
        \choice $30-50\%$
        \choice $51-70\%$
        \choice $>70\%$
    \end{checkboxes}
\end{parts}
\addpoints

\question[8] \textbf{Beantworten} Sie die folgenden Fragen. 
\noaddpoints
\begin{parts}
    \part[2] \textbf{Erklären} Sie in Stichworten den Unterschied zwischen einem \textbf{präskriptiven} und einem \textbf{deskriptiven} Modell.
    \part[2] \textbf{Nennen} Sie \textbf{vier gebräuchliche Darstellungsformen für Modelle} im Software Engineering.
    \part[2] \textbf{Nennen} sie \textbf{vier Aspekte}, die durch den \textbf{Einsatz von Modellen verbessert} werden können.
    \part[2] Die Abbildung zeigt die Unterstützung einer riskanten Modifikation durch Modelle. \textbf{Ergänzen} sie die Abbildung um die fehlenden Begriffe. \\
    \begin{tikzpicture}[node distance = 3 cm]
        \node[place] (00) {}
            node[xshift = -20mm] {Ist-Zustand};
        \node[place, right = of 00] (01) {}
            node[xshift = 70mm, yshift = -3mm]{\underline{\hspace{40mm}}};
        \node[place, below = of 00] (10) {}
            node[xshift = -20mm, yshift = -40mm, align = center] {Geplanter \\Zustand};
        \node[place, right = of 10] (11) {}
            node[xshift = 70mm, yshift = -43mm] {\underline{\hspace{40mm}}};
        
        \draw[->] (00) -- (01)
            node[midway, fill = white, yshift = -4mm] {Modellierung};
        \draw[->] (00) -- (10)
            node[midway, fill = white, align = center, xshift = -15mm] {Riskante \\Modifikation};
        \draw[->] (01) -- (11)
            node[midway, xshift = 30mm, yshift = -3mm] {\underline{\hspace{40mm}}};
        \draw[->] (11) -- (10);

    \end{tikzpicture}
\end{parts}
\addpoints

\question[3] Softwaregestützte Systeme können nach Anwendungsgebiet grob in drei Klassen (z. B. "`technische Systeme"') eingeordnet werden. \textbf{Ordnen} Sie die folgenden Beispiele jeweils der \textbf{richtigen Klasse} zu.
\addpoints
{
\checkboxchar{\underline{\hspace{4cm}}}
\begin{checkboxes}
\choice Bubblesort
\choice Flugleitsystem
\choice Wettervorhersage
\choice Autopilot
\choice Tetris
\choice Facebook
\end{checkboxes}
}

\question[2] \textbf{Nennen} Sie \textbf{zwei Wechselwirkungen} zwischen System und Systemumgebung.
\addpoints

\question[3] Tragen Sie in die Abbildung die \textbf{Bezeichnungen der Phasen des System Engineering-Prozesses} ein. \\
\addpoints
\begin{tikzpicture}[transition/.style={minimum height = 1.5cm, minimum width = 3cm, draw = black}]
    \node[transition] (00) {};
    \node[transition, xshift = 4cm, yshift = 0cm] (01) {};
    \node[transition, xshift = 8cm, yshift = 0cm] (02) {};
    \node[transition, xshift = 12cm, yshift = 0cm] (03) {};
    
    \node[transition, xshift = 4cm, yshift = -2.5cm] (11) {Ausmusterung};
    \node[transition, xshift = 8cm, yshift = -2.5cm] (12) {};
    \node[transition, xshift = 12cm, yshift = -2.5cm] (13) {};

    \draw[->] (00) -- (01);
    \draw[->] (01) -- (02);
    \draw[->] (02) -- (03);
    \draw[->] (03) -- (13);
    \draw[->] (13) -- (12);
    \draw[->] (12) -- (11);
\end{tikzpicture}

\question[2] Struktur und Merkmale von im Prozess benötigten Dokumenten gehören zu den Inhalten eines Prozessmodells. \textbf{Nennen} Sie \textbf{vier weitere Inhalte} eines Prozessmodells.
\addpoints

\question[3] Zu einem Prozessmodell gehören beispielsweise Prüf- und Fertigstellungskriterien. \textbf{Nennen} Sie \textbf{drei weitere Inhalte} eines Prozessmodells.
\addpoints

\question[2] Ordnen Sie das \textbf{Wasserfallmodell} ein, indem Sie die fehlenden Elemente in der Abbildung ergänzen. \\
\addpoints
\begin{tikzpicture}[grow = right,
                    level 1/.style={sibling distance = 6em},
                    level 2/.style={sibling distance = 3em},
                    level distance = 4cm,
                    transition/.style={text width = 3cm, minimum height = 1cm, draw = black}]
    \node[transition]{Vorgehens- und Prozessmodelle}
    child{node[transition]{}}
    child{node[transition]{}}
    child{node[transition]{}
        child{node[transition]{}}
        child{node[transition]{}
            child{node[transition]{}}
        }
    };
\end{tikzpicture}

\question[2] Nennen sie zwei nicht-lineare Prozessmodelle.
\addpoints

\question[3] \textbf{Nennen} Sie \textbf{zwei Vorteile} und \textbf{einen Nachteil} des \textbf{klassischen Wasserfallmodells}.
\addpoints
{
\checkboxchar{ }
\begin{checkboxes}
    \choice \textbf{+}
    \choice \textbf{+}
    \choice \textbf{-}
\end{checkboxes}
}

\question[4] \textbf{Erläutern} Sie die Begriffe \textbf{Verifikation} und \textbf{Validation}.
\addpoints

\question[2] Systemerstellung (SE)  ist ein Submodell des V-Modell 97. Nennen Sie zwei weitere Submodelle des V-Modells 97.
\addpoints

\question[2] Ordnen Sie das V-Modell 97 in den \textbf{Kontext der Vorgehens- und Prozessmodelle} ein.
\addpoints

\question[2] Geben sie durch Ankreuzen des jeweiligen Feldes an, ob die folgenden Aussagen wahr oder falsch sind.
\addpoints
\begin{tabular}{| p{12cm} | c | c |}
    Zu den Submodellen des V-Modells geh"ort das Werkzeugmanagement. & & \\ \hline
    Das Spiralmodell ist ein generisches Modell f"ur Projektmanagement. & & \\ \hline
\end{tabular}

\question[3] Die Abbildung zeigt die Aufwandsverteilung in Abh"angigkeit der Projektgr"o"se. \\
\noaddpoints
\begin{tikzpicture}
    \begin{scope}[xshift = 1cm]
        \draw (0, 0) -- (8, 0) -- (8, 2) -- (0, 1) -- cycle;
        \draw (0, 1) -- (8, 2) node[xshift = -2cm, yshift = 1.5cm] {Dokumentation} -- (8, 2.5) -- (0, 4) -- cycle;
        \draw (0, 4) -- (8, 2.5) -- (8, 4) -- (0, 4.5) -- cycle;
        \draw (0, 4.5) -- (8, 4) -- (8, 5) -- (0, 5) -- cycle;
    \end{scope}

    %TODO: x axis blabla
    \foreach \y/\ytext in {0/$0\%$, .5/$10\%$, 1/$20\%$, 1.5/$30\%$, 2/$40\%$, 2.5/$50\%$, 3/$60\%$, 3.5/$70\%$, 4/$80\%$, 4.5/$90\%$, 5/$100\%$}
        \node[yshift = \y cm, font = \footnotesize] {\ytext};
\end{tikzpicture}
\begin{parts}
    \part[1.5] Tragen Sie die fehlenden Aktivit"aten \textbf{Management, Codierung und Qualit"atssicherung} in die Abbildung ein.
    \part[1.5] Begr"unden Sie in \textbf{ganzen S"atzen}, warum sich die Aufwandsverteilung so verh"alt.
\end{parts}
\addpoints

\question[6] Das \textbf{V-Modell 97} legt Aktivit"aten, Produkte und Zust"ande des Entwicklungsprozesses fest. Tragen sie die \textbf{Aktivit"aten} des \textbf{Submodells f"ur Systemerstellung (SE)} in die Abbildung ein.\\
\addpoints
\begin{tikzpicture}[rectangle/.style={draw = black, fill = white, minimum width = 3cm, minimum height = 1cm}]
    \draw (0, 0) -- (5, -10) -- (10, 0);
    \foreach \xshift/\yshift in {0, 1/2, 2/4, 3/6, 4/8, 5/10, 10/0, 9/2, 8/4}
        \node[xshift = \xshift cm, yshift = -\yshift cm, rectangle] {};
\end{tikzpicture}

\question[10]
\noaddpoints
\begin{parts}
    \part[6] Vergleichen Sie das \textbf{V-Modell 97} mit dem Prozessmodell \textbf{Prototyping}. Beziehen Sie in der Verglech auch die Vor- und Nachteile beider Modelle mit ein.
    \part[4] Nennen Sie \textbf{zwei weitere Prozessmodelle} und beschreiben Sie jedes in ein bis zwei S"atzen.
\end{parts}
\addpoints

\question[4] Geben Sie durch Ankreuzen des jeweiligen Antwortfeldes an, ob die folgenden Aussagen wahr oder falsch sind. \\
\addpoints
\begin{tabular}{| p{12cm} | c | c |} \hline
    \textbf{Aussage} & \textbf{wahr} & \textbf{falsch} \\ \hline
    Beim horizontalen Prototyp werden ausgew"ahlte Teile des Systems durch alle horizontalen Schichten realisiert. & & \\ \hline
    MDA steht auch f"ur Model Driven Architecture. & & \\ \hline
    Aus Modellen generierter Code erf"ullt die Anforderungen stets ohne weitere Pr"ufung. & & \\ \hline
    Code and Fix ist auch ein Prozessmodell. & & \\ \hline
\end{tabular}

\question[3] \textbf{Erg"anzen} Sie folgende Aussagen des \textbf{Agilen Manifests}.
\addpoints
\begin{parts}
    \part \underline{\hspace{5cm}} sind wichtiger als Prozesse und Werkzeuge.
    \part \underline{\hspace{5cm}} sind wichter als ausf"uhrliche Dokumentation.
    \part \underline{\hspace{5cm}} ist wichtiger, als einem Plan zu folgen.
\end{parts}

\question[2] Erg"anzen Sie zwei der folgenden drei S"atze aus dem \textbf{Agilen Manifest von 2001}.
\addpoints
\begin{parts}
    \part Individuen und Interaktionen sind wichtiger als \underline{\hspace{5cm}}.
    \part \underline{\hspace{5cm}} sind wichtiger als ausf"uhrliche Dokumentation.
    \part Die Zusammenarbeit mit dem Kunden ist wichtiger als \underline{\hspace{5cm}}.
\end{parts}

\question[2] Erkl"aren Sie kurz den Unterschied zwischen \textbf{explorativem} und \textbf{evolution"arem Prototyping}.
\addpoints

\question[4] Was ist der \textbf{Unterschied} zwischen \textbf{explorativem} und \textbf{evolution"arem Prototyping}? Wann wird welcher Typ eingesetzt?
\addpoints

{
\checkboxchar{ }
\question \textbf{Nennen} Sie \textbf{zwei Vorteile} und \textbf{zwei Nachteile} des \textbf{Prototyping}.
\addpoints
\begin{checkboxes}
    \choice \textbf{+}
    \choice \textbf{+}
    \choice \textbf{-}
    \choice \textbf{-}
\end{checkboxes}
}

\question[4] Erkl"aren Sie die folgenden Begriffe und nennen sie jeweils ein Beispiel.
\addpoints
\begin{parts}
    \part Wegwerf-Prototyp
    \part Evolution"arer Prototyp
    \part Horizontaler Prototyp
    \part Vertikaler Prototyp
\end{parts}

\question[3] Erg"anzen Sie die fehlenden Phasen des V-Modells in der Abbildung. \\
\addpoints
\begin{tikzpicture}[rectangle/.style={draw = black, fill = white, minimum width = 3cm, minimum height = 1cm}]
    \draw (0, 0) -- (5, -10) -- (10, 0);
    \foreach \xshift/\yshift in {0, 1/2, 3/6, 4/8, 5/10, 8/4}
        \node[xshift = \xshift cm, yshift = -\yshift cm, rectangle] {};
    \node[xshift = 2cm, yshift = -4cm, rectangle] {SW-/HW-Anforderungsanalyse};
    \node[xshift = 10cm, rectangle] {"Uberleitung in die Nutzung};
    \node[xshift = 9cm, yshift = -2cm, rectangle] {System-Integration};
\end{tikzpicture}

\question[4] Nennen sie zwei gute und zwei schlechte Kriterien f"ur das Ende von Pr"ufaktivit"aten.
\addpoints

\question[5] In zwei unabh"angigen Pr"ufvorg"angen des selben St"uck Software wurden einmal 100 Fehler und einmal 60 Fehler gefunden. Davon wurden 30 Fehler in beiden Pr"ufvorg"angen gefunden.
\noaddpoints
\begin{parts}
    \part[1] Wie nennt man das Verfahren, mit dem aus den oben gegebenen Informationen die Gesamtfehlerzahl gesch"atzt werden kann?
    \part[2] Wie hoch wird die Gesamtfehlerzahl f"ur das gegebene Beispiel mit dieser Methode gesch"atzt?
    \part[2] Unter welchen Umst"anden sollte diese Methode besser nicht angewendet werden? Nennen sie zwei F"alle.
\end{parts}

\question[2] Bei einem Review mit zwei Gutachtern findet der eine Gutachter 147 Fehler, der andere 122 davon verschiedene Fehler. Enth"alt die Software vermutlich weitere Fehler? Begr"unden Sie Ihre Antwort.
\addpoints

\question[5] Von einem St"uck Software wissen wir, dass es 20 echte Fehler enth"alt. Nach der Durchf"uhrung einer Restfehlersch"atzung mittels error seeding haben die Pr"ufer von den 15 injizierten Fehlern drei gefunden. Wie viele echte Fehler wurden zur Restfehlersch"atzung injiziert? Welche Schw"ache hat dieses Verfahren der Restfehlersch"atzung?
\addpoints

\question[5] Von einem St"uck Software wissen wir, dass es 40 echte Fehler enth"alt. Nach der Durchf"uhrung einer Restfehlersch"atzung mittels error seeding haben die Pr"ufer acht echte Fehler und f"unf injizierte Fehler gefunden. Wie viele Fehler wurden zur Restfehlersch"atzung injiziert? Welche Schw"ache hat dieses Verfahren der Restfehlersch"atzung?
\addpoints

\question[3] Erkl"aren Sie die Idee der Restfehlersch"atzung durch Error Seeding und leiten Sie die entsprechende Berechnungsformel ab.
\addpoints

\question[4] Erkl"aren Sie die zugrundeliegende Idee der Defensiven Programmierung und Redundanten Programmierung. Erl"autern Sie auch die entsprechenden Umsetzungsm"oglichkeiten.
\addpoints

\question[6] Geben sie durch Ankreuzen eines der Antwortfelder an, ob die folgenden Aussagen wahr oder falsch sind. \\
\addpoints
\begin{tabular}{| p{12cm} | c | c |} \hline
    Aussage & wahr & falsch \\ \hline
    Eine dynamische Pr"ufung kann im V-Modell zu jedem Zeitpunkt durchgef"uhrt werden. & & \\ \hline
    Der letzte Schritt eines Reviews ist stets die Nacharbeit. & & \\ \hline
    Der Vorteil von Reviews ist, dass sie bei hoher Wirksamkeit nur wenig Kosten verursachen. & & \\ \hline
    Je kleiner das Vorhaben ist, desto gr"o"ser sollte der Anteil der Arbeitsergebnisse sein, der durch Reviews gepr"uft wird. & & \\ \hline
    Die Erhebung von Metriken ist kein systematischer Test. & & \\ \hline
    Ein Audit kann ein Review ersetzen. & & \\ \hline
\end{tabular}

\question[6] Geben sie durch Ankreuzen eines der Antwortfelder an, ob die folgenden Aussagen wahr oder falsch sind. \\
\addpoints
\begin{tabular}{| p{12cm} | c | c |} \hline
    Aussage & wahr & falsch \\ \hline
    Der Glasbox-Test eignet sich gut zum Testen ganzer Systeme. & & \\ \hline
    Im Schnitt sieben Fehler pro 1000 Programmzeilen sind normal f"ur sicherheitskritische Software. & & \\ \hline
    Testen ist eine typische konstruktive Qualit"atssicherungsma"snahme. & & \\ \hline
    Beim systematischen Testen geht es darum, die Korrektheit der Software nachzuweisen. & & \\ \hline
    An Dokumenten f"ur ein technisches Review braucht man neben dem Pr"ufling nur noch Referenzunterlagen (zum Beispiel Spezifikation, Richtlinien, Fragenkataloge, Standards). & & \\ \hline
    Der Einsatz eines Vorgehensmodells ist eine analytische Qualit"atssicherungsma"snahme. & & \\ \hline
\end{tabular}

\question[8] Geben sie durch Ankreuzen eines der Antwortfelder an, ob die folgenden Aussagen wahr oder falsch sind. \\
\addpoints
\begin{tabular}{| p{12cm} | c | c |} \hline
    Aussage & Wahr & Falsch \\ \hline
    Ein Systemtest testet das gesamte Softwaresystem in der realen Umgebung. & & \\ \hline
    Zwei Testf"alle sind hinsichtlich des testerfolgs stark "aquivalent, wenn beide geeignet sind, einen bestimmten Fehler anzuzeigen. & & \\ \hline
    Die Pfad"uberdeckung ist zu 100\% erf"ullt, sobald die Zweig"uberdeckung zu 100\% erf"ullt ist. & & \\ \hline
    Mutationstests sind strukturelle Tests. & & \\ \hline
    Der Zweck einer statischen Pr"ufung ist f"u die Pr"ufung irrelevant. & & \\ \hline
    Der letzte Schritt eines Reviews ist stets die Nacharbeit. & & \\ \hline
    Bei einer Restfehlersch"atzung mittels error seeding wurden 25 Fehler injeziert. Die Pr"ufer haben 8 echte Fehler und 5 injuzierte Fehler gefunden. Daraus l"asst sich schlie"sen, dass 40 echte Fehler existieren. & & \\ \hline
    Die Erhebung von Metriken ist ein systematischer Test. & & \\ \hline1
\end{tabular}

\question[8] Geben sie durch Ankreuzen eines der antwortfelder an, ob die folgenden Aussagen wahr oder falsch sind. \\
\addpoints
\begin{tabular}{| p{12cm} | c | c |} \hline
    Aussage & Wahr & Falsch \\ \hline
    "Aquivalenzklassen beim Blackbox-Test sind Teilbereiche des eingabebereichs einer funktion, die sich bez"uglich ihres Funktionswerts gleich verhalten. & & \\ \hline
    Die Zyklomatische Komplexit"at basiert auf der Annahme, dass die Komplexit"at eines Programms von der Anzahl der Anweisungen abh"angt. & & \\ \hline
    Die L"ange des Benutzerhandbuchs ist ein Softwareattribut. & & \\ \hline
    Es ist generell m"oglich, bei einem Glassbox-Test immer alle vorhandenen Pfade auszuf"uhren. & & \\ \hline
    Ein guter Test beginnt mit der Planungsphase. & & \\ \hline
    Unter Plausibilit"at einer Metrik versteht man, dass die empirische Bewertung der Ma"szahl mit den Beobachtungen korelliert. & & \\ \hline
    Inspektion ist eine Form des Reviews. & & \\ \hline
    Bei einem Review wird neben dem Pr"ufling auch der Autor begutachtet. & & \\ \hline
\end{tabular}

\question[5] Nennen sie f"ur die Phasen Anforerungsanalyse und Entwurf
\noaddpoints
\begin{parts}
    \part[2] den Zusammenhand beider Phasen.
    \part[1] eine Gemeinsamkeit beider Phasen.
    \part[2] zwei Unterschiede beider Phasen.
\end{parts}
\addpoints

\question[2] Nennen sie zwei Aspekte, die beim entwurd verglichen mit der Analyse zus"atzlich ber"ucksichtigt werden m"ussen.
\addpoints

\question[4] Nennen sie vier Entwurdskonzepte.
\addpoints

\question[6] Modularit"at ist eines der wichtigsten Entwurfskonzepte. Nennen und erl"autern sie drei weitere wichtige Entwurfskonzepte.
\addpoints

\question[3] Nennen sie je ein Beispiel f"ur funktionale Abstraktion, Datenabstraktion und Kotrollabstraktion.
\addpoints

\question[2] Nennen sie zwei positive Effekte durch eine hohe Modularit"at beim Software-Entwurf?
\addpoints

\question[4] Ein guter objektorientierter Entwurd hat einen starken logischen Zusammenhang innerhalb einer Komponente und pr"azise definierte, minimale Schnittstellen zwischen den Komponenten. Wie hei"sen die beiden zugeh"origen Qualit"atskriterien? Geben sie die Formel an, die aus diesen beiden Kriterien das Optimum berechnet.
\addpoints

\question[2] Erkl"aren sie kurz die Begriffe Koh"asion und Kopplung.
\addpoints

\question[4] Welcher Zusammenhang besteht zwischen Koh"asion und Kopplung und der Qualit"at eines Software-Entwurfs?
\addpoints

\question[8] Vervollst"andigen Sie das UML-Klassendiagramm in der Abbildung so, dass es folgenden Sachverhalt als Dom"anenmodell beschreibt. Vergessen Sie nicht, auch die Kardinalit"aten anzugeben! \\
\emph{Tutorien an der Universit"at Ulm \\Ein Tutor hat einen Namen, eine E-Mail-Adresse und eine Adresse. Er oder die h"alt mindestens ein Tutorium. Kedes Tutorium wird von genau einem tutor geleitet. Ein Student (mit Name, Matrikelnummer und E-Mail) nimmt an genau einem Tutorium teil. An einem Tutorium k"onnen maximal 20 Studenten teilnehmen. Es gibt keine Tutorien ohne Teilnehmer. Tutorien finden an einem oder mehreren Terminen (mit Datum, Uhrzeit und Dauer) statt. Da mehrere R"aume zur Verf"ugung stehen, k"onnen Tutorien auch parallel stattfinden. Ein Stundenplan besteht aus beliebig vielen Terminen. Ein Termin kann auch in mehreren Stundenpl"anen auftauchen, muss aber in mindestens einem vorhanden sein. Auch leere Stundenpl"ane sind m"oglich. Einem Termin ist ein Raum zugeordnet. R"aume haben eine Raumbezeichung.} \\
\addpoints
\begin{tikzpicture}[title/.style={rectangle, minimum height = 2cm, minimum width = 3cm, draw = black}]
    \matrix[column sep = 1.5cm, row sep = 1cm]{
        %1st row
        &
        &
        \node[title] (02) {}; \\
        %2nd row
        \node[title] (10) {}; &
        \node[title] (11) {}; &
        \node[title] (12) {}; \\
        %3rd row
        &
        \node[title] (21) {}; &
        \node[title] (22) {}; \\
    };
    
    \draw (02) -- (12);
    \draw (10) -- (11);
    \draw (11) -- (12);
    \draw (11) -- (21);
    \draw (12) -- (22);
\end{tikzpicture}

\question[5] F"ur ein System sind folgende Anforderungen gegeben:
\begin{itemize}
    \item \emph{Das System verwaltet Pl"atze, Reservierungen, S"ale und Vorstellungen.}
    \item \emph{Eine Reservierung besteht aus mehreren Pl"atzen (mindestens einem) und geh"ort zu genau einer Vorstellung.}
    \item \emph{Zu einer vorstellung kann es mehrere (aber auch keine) Reservierung/-en geben.}
    \item \emph{Eine Vorstellung findet in genau einem Saal statt. Ein Platz geh"ort zu einem Saal. Ein Saal besteht aus mindestens 10 Pl"atzen.}
    \item \emph{In einem Saal k"onnen beliebig viele Vorstellungen stattfinden.}
\end{itemize}
Erg"anzen Sie das Klassendiagramm in der Abbildung um die Kardinalit"aten, sodass die Anforderungen erf"ullt sind. Fehlende Angaben m"ussen Sie sinnvoll erg"anzen. Welches Problem entsteht und wie k"onnte man es l"osen? \\
\addpoints
\begin{tikzpicture}[rectangle/.style = {minimum height = 1.5cm, minimum width = 3cm, draw = black}]
    \matrix[column sep = 3cm, row sep = 3cm]{
        %1st row
        \node[rectangle] (00) {}; &
        \node[rectangle] (01) {}; \\
        %2nd row
        \node[rectangle] (10) {}; &
        \node[rectangle] (11) {}; \\
    };
    \draw (00) -- (01);
    \draw (00) -- (10);
    \draw (01) -- (11);
    \draw (10) -- (11);
\end{tikzpicture}

\question[5] Nennen Sie f"ur die Phasen Anforderungsanalyse und Entwurf
\noaddpoints
\begin{parts}
    \part[2] den Zusammenhang beider Phasen.
    \part[1] eine Gemeinsamkeit beider Phasen.
    \part[2] zwei Unterschiede beider Phasen.
\end{parts}

\question[2] Nennen sie zwei Aspekte, die beim Entwurf verglichen mit der Analyse zus"atzlich ber"ucksichtigt werden m"ussen.
\addpoints

\question[4] Nennen sie vier Entwurfskonzepte.
\addpoints

\question[6] Modularit"at ist eines der wichtigsten Entwurfskonzepte. Nennen und erl"autern sie drei weitere wichtige Entwurfskonzepte.
\addpoints

\question[3] Nennen sie je ein Beispiel f"ur funktionale Abstraktion, Datenbankabstraktion und Kontrollabstraktion.
\addpoints

\question[2] Nennen sie zwei positive Effekte durch hohe Modularit"at beim Software-Entwurf.
\addpoints

\question[4] Ein guter objektorientierter Entwurd hat einen starken logischen Zusammenhang innerhalb einer Komponente und pr"azise definierte, minimale Schnittstellen zwischen den Komponenten. Wie hei"sen die beiden zugeh"origen Qualit"atskriterien? Geben sie die Formel an, die aus diesen beiden Kriterien das Optimum berechnet.
\addpoints

\question[2] Erkl"aren sie kurz die Begriffe Koh"asion und Kopplung.
\addpoints

\question[4] Welcher Zusammenhang besteht zwischen Koh"asion und Kopplung und der Qualit"at eines Software-Entwurfs?
\addpoints

\question[2] Die Abbildung zeigt zwei Entwurfsskizzen. Die Rechtecke symbolisieren die Module, die Verbindungen die Kommunikationswege. Welche der beiden Entwurfsskizzen ist unter dem Aspekt der Kopplung besser zu bewerten? Begr"unden sie ihre Antwort kurz.\\
\addpoints
\begin{tikzpicture}[scale = .5]
    \begin{scope}[rectangle/.style={draw = black, text width = 1cm, minimum height = 5mm}]
        \node (0) at (2, 0) [rectangle] {};
        \node (1) at (6, 0) [rectangle] {};
        \node (2) at (8, 4) [rectangle] {};
        \node (3) at (4, 7) [rectangle] {};
        \node (4) at (0, 4) [rectangle] {};

        \draw (0) -- (1);
        \draw (0) -- (3);
        \draw (1) -- (2);
        \draw (1) -- (3);
        \draw (2) -- (3);
        \draw (2) -- (4);
        \draw (3) -- (4);
        \draw (4) -- (0);
    \end{scope}

    \begin{scope}[xshift = 14cm, rectangle/.style={draw = black, text width = 1cm, minimum height = 5mm}]
        \node (0) at (2, 0) [rectangle] {};
        \node (1) at (6, 0) [rectangle] {};
        \node (2) at (8, 4) [rectangle] {};
        \node (3) at (4, 7) [rectangle] {};
        \node (4) at (0, 4) [rectangle] {};
        \node (5) at (4, 4) [rectangle] {};

        \draw (0) -- (5);
        \draw (1) -- (5);
        \draw (2) -- (5);
        \draw (3) -- (5);
        \draw (4) -- (5);
    \end{scope}
\end{tikzpicture}

\question[4] Sind folgende Aussagen wahr oder falsch? Begr"unden sie ihre Antwort.
\addpoints
\begin{parts}
    \part Auf den Feinentwurf kann man verzichten, auf den Implementierungsentwurf jedoch nicht.
    \part Aufgrund der Struktur in einem DFD kann auf ein Transaktionszentrum geschlossen werden.
    \part Eine gute Modularisierung ist eine wichtige Voraussetzung.
    \part Ein Vorteil des Repositorymodells ist die Effizienz f"ur gro"se Datenbest"ande.
\end{parts}

\question[6] Nennen und beschreiben sie zwei der aus der Vorlesung bekannten Systemstrukturen. Nennen sie zu jedem jeweils zwei Vor- und Nachteile.
\addpoints

\question[4] Erkl"aren sie die Steuerstruktur 'Zentralistische Steuerung' anhand zweier m"oglicher Auspr"agungen.
\addpoints

\question[2] Nennen sie je eine m"ogliche Auspr"agung f"ur die Kontrollmodelle 'Zentralistische Steuerung' und 'Ereignisbasierte Steuerung'.
\addpoints

\question[4] Die modulare Zerlegung beschreibt die Zerlegung von Subsystemen in Module. Nennen sie zwei M"oglichkeiten der modularen Zerlegung und erl"autern sie diese kurz.
\addpoints

\question[4] Erkl"aren sie den Unterschied zwischen ereignisbasierter und zeitbasierter Steuerung im Rahmen des Architekturenentwurfs. In welchem Kontext sollte welches der beiden Kontrollmodelle eingesetzt werden?
\addpoints

\question[4] Nennen sie jeweils zwei Vor- und Nachteile des Client-Server-Modells.
\addpoints

\question[6] Geben sie durch Ankreuzen eines der Antwortfelder an, ob die folgenden Aussagen wahr oder falsch sind. \\
\addpoints
\begin{tabular}{| p{12cm} | l | l |} \hline
    Aussage & Wahr & Falsch \\ \hline
    Ein Vorteil des Repositorymodells ist die Effizient f"ur gro"se Datenbest"ande. & & \\ \hline
    Abstract-Machine-Modell wird auch "'Schichtenmodell"` genannt. & & \\ \hline
    Der strukturierte Entwurf verwendet das Prinzip des Bottom-Up-Vorgehens. & & \\ \hline
    Funktionsorientierte Entw"urfe eignen sich besonders f"ur Systeme mit komplexen Datenstrukturen. & & \\ \hline
    Auf den Feinentwurf kann man verzichten, auf den Implementierungsentwurf nicht. & & \\ \hline
    Eine gute Modularisierung ist eine wichtige Voraussetzung f"ur eine arbeitsteilige Implementierung. & & \\ \hline
\end{tabular}

\question[3] Sie bekommen den Auftrag, die Benutzerschnittstelle f"ur ein neues System zu entwerfen.
\noaddpoints
\begin{parts}
    \part[1.5] Welche drei grunds"atzlichen Aspekte m"ussen sie dabei ber"ucksichtigen?
    \part[1.5] Nennen sie drei Punkte, die sie kl"aren m"ussen, um eine gute Benutzerschnittstelle zu entwerfen.
\end{parts}

\question[3] Geben sie durch Ankreuzen eines der Antwortfelder an, ob die folgenden Aussagen wahr oder falsch sind. \\
\addpoints
\begin{tabular}{| p{12cm} | l | l |} \hline
    Aussage & Wahr & Falsch \\ \hline
    Pair-Programming ist eine Auspr"agung des Ego-less-Programming. & & \\ \hline
    Bei einer guten Kommentierung wird der Programmtext durch Umgangssprache paraphrasiert. & & \\ \hline
    Mit dem Begriff meta CASE werden Werkzeuge bezeichnet, die andere Werkzeuge beinhalten, zusammenfassen oder einfach nur benutzen. & & \\ \hline
\end{tabular}

\question[2] Was ist bei Dokumentation grunds"atzlich zu beachten?
\addpoints

\question[2] Eine Grundvoraussetzung f"ur grunds"atzliche Dokumentation in einem Projekt ist, dass Verantwortlichkeiten und Anforderungen an die Dokumentation klar geregelt sind. Nennen sie vier weitere Grundvoraussetzungen f"ur zus"atzliche Dokumentationen.
\addpoints

\question[2] Nennen sie zwei Aspekte, die grunds"atzlich bei der zus"atzlichen Dokumentation beachtet werden sollen.
\addpoints

\question[3] Welche Aspekte hinsichtlich Werkzeuganforderungen m"ussen bei der Werkzeugwahl ber"ucksichtigt werden? Nennen sie vier Aspekte und erl"autern sie zwei davon anhand von Stichworten.
\addpoints

\question[3] Nennen sie drei Kriterien zur Klassifikation von Werkzeugen.
\addpoints

\question[4] Nennen sie vier Probleme bei der Auswahl von Werkzeugen f"ur die Softwareerstellung.
\addpoints

\question[3] Geben sie durch Ankreuzen eines der Antwortfelder an, ob die folgenden Aussagen wahr oder falsch sind. \\
\addpoints
\begin{tabular}{| p{12cm} | l | l |} \hline
    Aussage & Wahr & Falsch \\ \hline
    Wichtig bei Programmierrichtlinien ist nicht, dass etwas geregelt wird, sondern wie genau etwas geregelt wird. & & \\ \hline
    Rund 90\% aller Programmierer sind mit der Wartung und Verbesserung von altem Code besch"aftigt. & & \\ \hline
    Mit dem Begriff upper CASE werden Werkzeuge bezeichnet, die in den fr"uhen Phasen der Entwicklung eingesetzt werden. & & \\ \hline
\end{tabular}

\question[6] Nennen sie zwei Beispiele f"ur Qualit"atsmerkmale von Software und zwei Beispiele f"ur Softwareattribute und erl"autern sie, wie diese zusammenh"angen.
\addpoints

\question[4] Differenziertheit und Vergleichbarkeit sind zwei wichtige Eigenschaften von Metriken. Nennen sie vier weitere wichtige Eigenschaften von Metriken und erkl"aren sie zwei davon mit jeweils einem Satz.
\addpoints

\question[4] Um die Qualit"at der Codedokumentation zu messen, wird eine Metrik mit $Q = \frac{\text{Anzahl der Codezeilen}}{\text{Anzahl der Kommentarzeilen}}$ definiert, wobei ein h"oherer Wert $Q$ eine h"ohere Qualit"at anzeigt. Zeigen sie, dass diese Metrik nicht plausibel ist.
\addpoints

\question[4] Nennen sie vier wichtige Eigenschaften von Metriken.
\addpoints

\question[3] Nennen sie drei wichtige Eigenschaften von Metriken und erl"autern sie diese jeweils in einem Satz.
\addpoints

\question[2] Was wird unter einer Metrix im Bereich Softwaretechnik verstanden?
\addpoints

\question[2] Die Abbildung zeigt den Ablaufgraph eines Bubblesort-Algorithmus. Geben sie die zyklomatische Komplexit"at des Algorithmus an und erkl"aren sie kurz, wie sie zu dem Ergebnis gekommen sind. \\
\addpoints
\begin{tikzpicture}[circle/.style={draw = black, minimum size = 1cm}]
    \node[circle] (3) {$3$};
    \node[circle] (4) [below = of 3] {$4$};
    \node[circle] (6) [below = of 4,
                        label = right:do] {$6$};
    \node[circle] (7) [below = of 6] {$7$};

    \node[circle] (15) [below left = of 7] {$15$};
    \node[circle] (16) [below = of 15] {$16$};
    \node[circle] (blankleft) [below = of 16] {};

    \node[circle] (9) [below right = of 7,
                        label = right:if] {$9$};
    \node[circle] (1112) [below = of 9] {$11, 12$};
    \node[circle] (blankright) [below = of 1112] {};

    \draw[->] (3) to (4);
    \draw[->] (4) to (6);
    \draw[->] (6) to (7);
    
    \draw[->] (7) to
        node[fill = white] {else}
        (15);
    \draw[->] (15) to (16);
    \draw[->] (16) to (blankleft);
    \draw[->, bend left = 45] (16) to (4);
    
    \draw[->] (7) to
        node[fill = white] {$i < n-1$}
        (9);
    \draw[->] (9) to 
        node[fill = white] {$A[i] > A[i+1]$}
        (1112);
    \draw[->, bend right = 75] (9) to
        node[fill = white] {else}
        (blankright);
    \draw[->] (1112) to (blankright);
    \draw[->, bend right = 90] (blankright) to (7);
\end{tikzpicture}

\question[3] Welche drei Arten von Qualit"atssicherungsma"snahmen gibt es? Nennen sie heweils die Ma"snahmenklasse und erkl"aren sie diese in jeweils einem Satz.
\addpoints

\question[4] Die zyklomatische Komplexit"at (McCabe-Metrik) misst die Komplexit"at eines Programms.
\noaddpoints
\begin{parts}
    \part[3] Zeichnen sie den Ablaufgraphen zu
        \begin{lstlisting}
            do {
                A;
                if B then C; else D;
            } while E;
        \end{lstlisting}    
    \part[1] Berechnen sie anhand des oben erzeugten Ablaufgraphen die zyklomatische Komplexit"at.
\end{parts}
\addpoints

\question[7] Erl"autern sie den grunds"atzlichen Ablauf eines Tests, indem sie die Abbildung vervollst"andigen. Geben sie auch die jeweiligen Ergebnisse der Phasen an. \\
\addpoints
\begin{tikzpicture}[circle/.style={minimum size = 5mm, draw = black},
                    rectangle/.style={draw = black, minimum height = 1cm, text width = 3cm}]
    \matrix{
        %1st row
        &
        \node (erg0) {Ergebnis: \underline{\hspace{4cm}}}; &
        & \\
        %2nd row
        \node[rectangle] (rect0) {}; &
        \node[circle] (circ0) {}; &
        & \\
    };
\end{tikzpicture}

\begin{comment}
{%
% changing choice items style locally
\renewcommand*\thechoice{\arabic{choice}} 
\renewcommand*\choicelabel{\thechoice)}
%
\question[2] Element with $Z=92$ is:
\begin{multicols}{2}
\begin{choices}
\choice H
\choice O
\choice F
\choice S
\choice Ba
\choice Pb
\choice U
\choice Pu
\end{choices}
\end{multicols}
}%

\question[10]
In no more than one paragraph, explain why the earth is round.
\makeemptybox{2in}

\question[20]
Explain blah, blah\ldots
\makeemptybox{\fill}

\newpage

\question[20]
Explain blah, blah\ldots
\fillwithlines{\fill}

\newpage

\question[20]
Explain blah, blah\ldots
\fillwithdottedlines{8em}
\end{comment}

\end{questions}

\end{document}
